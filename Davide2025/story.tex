\documentclass[a4paper,11pt]{article}
\pdfoutput=1 
\usepackage{jheppub} 
\usepackage[numbers]{natbib}
\usepackage[T1]{fontenc} 
\usepackage[compat=1.1.0]{tikz-feynman}
\usepackage{xcolor}
\usepackage{multirow}
\usepackage{multicol}
\usepackage{float}
\usepackage{verbatim,slashed}
\usepackage{amsmath,mathtools}
\usepackage{relsize,float}
\usepackage{bm, nicefrac}
\usepackage{hyperref}
\usepackage[capitalise]{cleveref}
\usepackage{comment}
\usepackage{booktabs}
\usepackage[normalem]{ulem}
\usepackage{overpic}

\newcommand{\pdp}{\ensuremath{\phi^\dagger\phi}}
\renewcommand{\phi}{\ensuremath{\varphi}}
\newcommand{\sss}{\scriptscriptstyle}
\newcommand{\sst}{\scriptstyle}
\newcommand{\OO}{\ensuremath{\mathcal{O}}}
\newcommand{\QQ}{\ensuremath{\mathcal{Q}}}
\newcommand{\Op}[1]{\OO_{\sss #1}}
\newcommand{\Opp}[2]{\OO_{\sss #1}^{\sss #2}}
\newcommand{\Oppd}[2]{^{\dagger}\OO_{\sss #1}^{\sss #2}}
\newcommand{\Qp}[1]{\QQ_{\sss #1}}
\newcommand{\Qpp}[2]{\QQ_{\sss #1}^{\sss #2}}
\newcommand{\Qppd}[2]{^{\dagger}\QQ_{\sss #1}^{\sss #2}}
\newcommand{\lam}{\Lambda}
\newcommand{\cp}[1]{c_{\sss #1}}
\newcommand{\Cp}[1]{C_{\sss #1}}
\newcommand{\cpp}[2]{c_{\sss #1}^{\sss #2}}
\newcommand{\Cpp}[2]{C_{\sss #1}^{\sss #2}}
\def\lra#1{\overset{\text{\scriptsize$\leftrightarrow$}}{#1}}
%\newcommand{\sw}{s_{\sss W}}
%\newcommand{\cw}{c_{\sss W}}
\newcommand{\mw}{m_{\sss W}}
\newcommand{\mz}{m_{\sss Z}}
\newcommand{\mt}{m_{t}}
\newcommand{\ttbar}{t\bar{t}}
\newcommand{\ttZ}{t\bar{t}Z}
\newcommand{\mwz}{m_{\sss WZ}}
\newcommand{\gztr}{g^{\sss Z}_{t_{\sss R}}}
\newcommand{\gztl}{g^{\sss Z}_{t_{\sss L}}}
\newcommand{\gzbr}{g^{\sss Z}_{b_{\sss R}}}
\newcommand{\gzbl}{g^{\sss Z}_{b_{\sss L}}}
\newcommand{\gbtw}{g_{\sss btW}}
\newcommand{\gta}{g_{\sss t\gamma}}
\newcommand{\gwa}{g_{\sss W\gamma}}
\newcommand{\gwz}{g_{\sss WZ}}
\newcommand{\dimreg}{dimensional regularisation}
\newcommand{\hv}{'t Hooft-Veltman }

\newcommand{\SDKweak}{Sudakov$_\text{weak}$ }

\newcommand{\OctTr}{$\Opp{Qq}{3,8}$}
\newcommand{\OctSi}{$\Opp{Qq}{1,8}$}
\newcommand{\QuOct}{$\Opp{Qu}{8}$}
\newcommand{\tqOct}{$\Opp{tq}{8}$}
\newcommand{\QdOct}{$\Opp{Qd}{8}$}
\newcommand{\tuOct}{$\Opp{tu}{8}$}
\newcommand{\tdOct}{$\Opp{td}{8}$}
\newcommand{\TriSi}{$\Opp{Qq}{3,1}$}
\newcommand{\SiSi}{$\Opp{Qq}{1,1}$}
\newcommand{\QuSi}{$\Opp{Qu}{1}$}
\newcommand{\tqSi}{$\Opp{tq}{1}$}
\newcommand{\QdSi}{$\Opp{Qd}{1}$}
\newcommand{\tuSi}{$\Opp{tu}{1}$}
\newcommand{\tdSi}{$\Opp{td}{1}$}
\newcommand{\QQSi}{$\Opp{QQ}{1}$}
\newcommand{\QQOct}{$\Opp{QQ}{8}$}
\newcommand{\ttSi}{$\Opp{tt}{1}$}
\newcommand{\QtSi}{$\Opp{Qt}{1}$}
\newcommand{\QtOct}{$\Opp{Qt}{8}$}

%%%%%%%%%%%
\renewcommand\arraystretch{1.1}
%\setlength{\textfloatsep}{10pt plus 1.0pt minus 2.0pt}

%NOTE: \beq and \eeq will not work if using amsmath. 
%Use \beqn and \eeqn instead
\def\beq{\begin{equation}}
\def\eeq{\end{equation}}
\def\beqn{\begin{eqnarray}}
\def\eeqn{\end{eqnarray}}
\def\abs#1{\left|#1\right|}
\def\IN{{\small IN}}
\def\OUT{{\small OUT}}
\def\remove#1#2{#1\hspace{-#2truecm}\backslash}
\def\removeb#1#2#3{#1\hspace{-#2truecm}\backslash\hspace{-#3truecm}\slash}
\def\red{}
\def\xidistr#1{\left(\frac{1}{\xi}\right)_{\!#1}}
\def\pdistr#1#2{\left(\frac{1}{#1}\right)_{\!#2}}
\def\lxidistr#1{\left(\frac{\log\xi}{\xi}\right)_{\!#1}}
\def\lpdistr#1#2{\left(\frac{\log#1}{#1}\right)_{\!#2}}
\def\lppdistr#1#2{\left(\frac{\log\left(#1\right)}{#1}\right)_{\!#2}}
\def\ypdistr#1{\left(\frac{1}{1-y}\right)_{\!#1}}
\def\FKSeq#1{eq.~({\bf II}.#1)}
\def\MadFKSeq#1{eq.~({\bf I}.#1)}
\def\set#1#2#3{\left\{#1\right\}_{#2,#3}}
\def\setS#1#2#3#4{\left\{#1\right\}_{#2,#3}^{[#4]}}
\def\oset#1#2#3{\left(#1\right)_{#2,#3}}
\def\binomial#1#2{
\left(\!\!
\begin{array}{c}
#1\\
#2
\end{array}
\!\!\right)
}


\newcommand{\bqa}{\begin{eqnarray}}
\newcommand{\eqa}{\end{eqnarray}}
\chardef\MyArticleWithColor=\pdfcolorstackinit page direct{0 g}
\def\cmtVH#1{\emph{\pdfcolorstack\MyArticleWithColor push {1 0 0 rg} V.H. : #1 \pdfcolorstack\MyArticleWithColor pop}}
\def\cmtHS#1{\emph{\pdfcolorstack\MyArticleWithColor push {0 1 0 rg} H.S. : #1 \pdfcolorstack\MyArticleWithColor pop}}
\newcommand{\plaat}[3]{\raisebox{#3pt}{\epsfig{figure=#1.pdf,width=#2cm}}}
\def\cCode#1{\begin{lstlisting}[mathescape,basicstyle=\small
\ttfamily,frame=leftline,aboveskip=4mm,belowskip=4mm,xleftmargin=20pt,framexleftmargin=10pt,
numbers=none,framerule=2pt,abovecaptionskip=0.0mm,belowcaptionskip=3.5mm #1]}


%\newcommand\sss{\scriptscriptstyle}
\newcommand\mydot{\!\cdot\!}
\newcommand\ep{\epsilon}
\newcommand\half{\frac{1}{2}}
\newcommand\quarter{\frac{1}{4}}
\newcommand\as{\alpha_{\sss S}}
\newcommand\gs{g_{\sss S}}
\newcommand\aW{\alpha_{\sss W}}
\newcommand\gW{g_{\sss W}}
\newcommand\aem{\alpha}
\newcommand\alo{\as}
\newcommand\alt{\aem}
\newcommand{\tev}{\,\textrm{TeV}}
\newcommand{\gev}{\,\textrm{GeV}}
\newcommand{\mev}{\,\textrm{MeV}}
\newcommand\bQ{\bar{Q}}
\newcommand\bt{\bar{t}}
\newcommand\aNLO{{\sc\small MadGraph5\_aMC@NLO}}
\newcommand\aNLOs{{\sc\small MG5\_aMC}}
\newcommand\MadGraph{{\sc\small MadGraph}}
\newcommand{\mgatnlo}{\aNLOs\xspace}
\newcommand{\mg}{\aNLOs\xspace}
\newcommand\UFO{{\sc\small UFO}}
\newcommand\mf{{\sc\small MadFKS}}
\newcommand\ml{{\sc\small MadLoop}}
\newcommand\ct{{\sc\small CutTools}}
\newcommand\nin{{\sc\small Ninja}}
\newcommand\collier{{\sc\small Collier}}
\newcommand\IREGI{{\sc\small IREGI}}
\newcommand\aMCSusHi{{\sc\small aMCSusHi}}
\newcommand\SusHi{{\sc\small SusHi}}
\newcommand\mgamc{\aNLO}
\newcommand\HWpp{{\sc\small Herwig++}}
\newcommand\HWsv{{\sc\small Herwig7}}
\newcommand\PYe{{\sc\small Pythia8}}
\newcommand\HWs{{\sc\small Herwig6}}
\newcommand\HWsette{{\sc\small Herwig7}}
\newcommand\FJ{{\sc\small FastJet}}
\newcommand\lhapdfs{{\sc\small LHAPDF6}}
\newcommand\recola{{\sc\small RECOLA}}
\newcommand\sherpa{{\sc\small Sherpa}}
\newcommand\alpgen{{\sc\small AlpGen}}
\newcommand\gosam{{\sc\small GoSam}}
\newcommand\maddip{{\sc\small MadDipole}}
\newcommand\OL{{\sc\small OpenLoops}}
\newcommand{\FeynRules}{{\sc \small FeynRules}}
\newcommand{\feynrules}{\FeynRules}
\newcommand{\nloct}{{\sc \small NloCt}}
\newcommand\prompt{{\tt MG5\_aMC>}}
\newcommand\pt{p_{\sss T}}
\newcommand\pti{p_{{\sss T},i}}
\newcommand\kt{k_{\sss T}}
\newcommand{\Ht}{H_{\sss T}}
\newcommand{\LO}{{\rm LO}}
\newcommand{\LOi}{{\rm LO}_i}
\newcommand{\LOipo}{{\rm LO}_{i+1}}
\newcommand{\LOo}{{\rm LO}_1}
\newcommand{\LOt}{{\rm LO}_2}
\newcommand{\LOth}{{\rm LO}_3}
\newcommand{\LOf}{{\rm LO}_4}
\newcommand{\NLO}{{\rm NLO}}
\newcommand{\NLOi}{{\rm NLO}_i}
\newcommand{\NpLOi}{{\rm N}^p{\rm LO}_i}
\newcommand{\NLOipo}{{\rm NLO}_{i+1}}
\newcommand{\NLOo}{{\rm NLO}_1}
\newcommand{\NLOt}{{\rm NLO}_2}
\newcommand{\NLOth}{{\rm NLO}_3}
\newcommand{\NLOf}{{\rm NLO}_4}
\newcommand{\NLOfv}{{\rm NLO}_5}
\newcommand{\NLOgt}{{\rm NLO}_{\ge 2}}
\newcommand{\NLOgth}{{\rm NLO}_{\ge 3}}
\newcommand{\MSb}{\overline{\rm MS}}
\newcommand{\epUV}{\varepsilon_{\rm\sss UV}}
\newcommand{\bepUV}{\bar{\varepsilon}_{\rm\sss UV}}
\newcommand{\epIR}{\varepsilon_{\rm\sss IR}}
\newcommand{\OS}{{\rm OS}}
\newcommand{\BW}{{\rm BW}}
\newcommand{\stepf}{\Theta}
\newcommand\muF{\mu_{\sss F}}
\newcommand\muR{\mu_{\sss R}}
\newcommand{\ttv}{t\bar{t}V}
\newcommand{\ttV}{\ttv}
\newcommand{\nmax}{{\tt n_{\tt max}}}
\newcommand{\mmax}{{\tt m_{\tt max}}}
\newcommand\FKSpairs{{\cal P}_{\sss\rm FKS}}
\newcommand\ren{{\rm R}}
\newcommand\unren{{\rm U}}
\newcommand\oG{\overline{G}}
\newcommand\vertmuCM{{\cal V}^{(0)\mu}_{\rm CM}}
\newcommand\vertnuCM{{\cal V}^{(0)\nu}_{\rm CM}}
\newcommand\vertmuZW{{\cal V}^{(0)\mu}_{\rm ZW}}
\newcommand\vertnuZW{{\cal V}^{(0)\nu}_{\rm ZW}}
\newcommand\bM{\bar{M}}
\newcommand\bGa{\bar{\Gamma}}
\newcommand\bga{\bar{\gamma}}
\newcommand\bMW{\bM_W}
\newcommand\bGaW{\bGa_W}
\newcommand\bMZ{\bM_Z}
\newcommand\bGaZ{\bGa_Z}
\newcommand\ZW{{\rm ZW}}
\newcommand\CM{{\rm CM}}


%%%%madqed.tex
\newcommand{\bq}{\bar{q}}
\newcommand{\epem}{e^+e^-}
\newcommand{\mpmm}{\mu^+\mu^-}
\newcommand{\ord}{{\cal O}}
\newcommand{\Sfun}{{\cal S}}
\newcommand{\Sfunij}{\Sfun_{ij}}
\newcommand\asotpi{\frac{\as}{2\pi}}
\newcommand\aotpi{\frac{\aem}{2\pi}}
\newcommand\muoQ{\frac{\mu^2}{Q^2}}
\newcommand\muoQep{\left(\frac{\mu^2}{Q^2}\right)^\ep}
\newcommand\Dz{{\cal D}^{(0)}}
\newcommand\Do{{\cal D}^{(1)}}
\newcommand\madfks{{\sc\small MadFKS}}
\newcommand\Tt{{\rm T}}
\newcommand\QCD{{\rm QCD}}
\newcommand\QED{{\rm QED}}
\newcommand\proc{r}
\newcommand\nini{n_{\sss I}}
\newcommand\nlight{n_{\sss L}}
\newcommand\nlightB{\nlight^{\sss (B)}}
\newcommand\nlightR{\nlight^{\sss (R)}}
\newcommand\nlightBorR{\nlight^{\sss (B/R)}}
\newcommand\nheavy{n_{\sss H}}
\newcommand\nzero{n_\emptyset}
\newcommand\ident{{\cal I}}
\newcommand\Ione{\ident_1}
\newcommand\Itwo{\ident_2}
\newcommand\amp{{\cal A}}
\newcommand\ampmt{\amp^{(m,0)}}
\newcommand\ampnt{\amp^{(n,0)}}
\newcommand\ampnpot{\amp^{(n+1,0)}}
\newcommand\ampnl{\amp^{(n,1)}}
\newcommand\ampsq{{\cal M}}
\newcommand\ampsqmt{\ampsq^{(m,0)}}
\newcommand\ampsqnt{\ampsq^{(n,0)}}
\newcommand\ampsqnpot{\ampsq^{(n+1,0)}}
\newcommand\ampsqnl{\ampsq^{(n,1)}}
\newcommand\vampsqnl{{\cal V}^{(n,1)}}
\newcommand\hvampsqnl{\hat{\cal V}^{(n,1)}}
\newcommand\vampsqnlF{{\cal V}^{(n,1)}_{\sss FIN}}
\newcommand\hvampsqnlF{\hat{\cal V}^{(n,1)}_{\sss FIN}}
\newcommand\tampsq{\widetilde{\cal M}}
\newcommand\tampsqnt{\tampsq^{(n,0)}}
\newcommand\tampsqnpot{\tampsq^{(n+1,0)}}
\newcommand\Qop{\vec{Q}}
\newcommand\Qops{Q}
\newcommand\JetsB{J^{\nlightB}}
\newcommand\JetsR{J^{\nlightB+1}}
\newcommand\velkl{v_{kl}}
\newcommand\avg{{\cal N}}
\newcommand\xicut{\xi_{cut}}
\newcommand\deltaO{\delta_{\sss O}}
\newcommand\deltaI{\delta_{\sss I}}
\newcommand\NC{N_{\sss c}}
\newcommand\CA{C_{\sss A}}
\newcommand\CF{C_{\sss F}}
\newcommand\TF{T_{\sss F}}
\newcommand\DA{D_{\sss A}}
\newcommand\nC{n_{\sss c}}
\newcommand\NF{N_{\sss F}}
\newcommand\Nl{N_l}
\newcommand\eikint{{\cal E}}
\newcommand\phsp{d\phi}
\newcommand\phspn{\phsp_{n}}
\newcommand\phspnpo{\phsp_{n+1}}
\newcommand\mua{\mu_\alpha}

\newcommand\ampzCM{\amp^{(0)}_{\rm CM}}
\newcommand\ampzZW{\amp^{(0)}_{\rm ZW}}



%%%%results_section.tex
\newcommand{\pnote}[1]{ \textbf{[DP:} \textit{\color{red} #1}\textbf{]}}
\newcommand{\TV}[1]{ \textbf{[TV:} \textbf{\color{blue} #1}\textbf{]}}
\newcommand{\LOone}{\ensuremath{\LOo}\xspace}
\newcommand{\LOtwo}{\ensuremath{\LOt}\xspace}
\newcommand{\LOthree}{\ensuremath{\LOth}\xspace}
\newcommand{\LOfour}{\ensuremath{\LOf}\xspace}
\newcommand{\NLOone}{\ensuremath{\NLOo}\xspace}
\newcommand{\NLOtwo}{\ensuremath{\NLOt}\xspace}
\newcommand{\NLOthree}{\ensuremath{\NLOth}\xspace}
\newcommand{\NLOfour}{\ensuremath{\NLOf}\xspace}
\newcommand{\NLOfive}{\ensuremath{\NLOfv}\xspace}
\newcommand{\NLOgetwo}{\ensuremath{\NLOgt}\xspace}
\newcommand{\NLOgethree}{\ensuremath{\NLOgth}\xspace}


\def\NLOEW{\rm NLO_{EW}}
\def\EWSL{\rm EWSL}
\def\LOQCDEWSL{\rm LO_{QCD+EWSL}}
\def\NLOQCDEWSL{\rm NLO_{QCD+EWSL}}

\def\NLOQCD{{\rm NLO_{QCD}}}
\def\NLOQCDPS{\rm NLO_{QCD}+PS} 
\def\LOQCD{{\rm LO_{QCD}}}
\def\LOQCDPS{\rm LO_{QCD}+PS} 
\def\NLOQCDt{\NLO_{{\rm QCD},t{\rm -ch.}}}
\def\NLOQCDEW{\NLO_{{\rm QCD+EW}}}
\def\NLOQCDEWPS{\NLOQCDEW+{\rm PS}}
\def\NLOQCDEWSLPS{\NLOQCDEWSL+{\rm PS}}
\def\LOQCDEWSLPS{\LOQCDEWSL+{\rm PS}}
\def\bestprednoPS{\NLOQCD\otimes{\rm EWSL}}
\def\bestpred{\bestprednoPS+{\rm PS}}
\def\PSnoQED{\rm PS_{\cancel{\rm QED}} }

\def\dEWSL{\delta^{\rm EWSL}}
\def\dEWSLH{\dEWSL_{\sss (\clH)}}
\def\dEWSLS{\dEWSL_{\sss (\clS)}}




%%%%%citation_summary.tex
%\newcommand{\mh}{m_{ \sss H}}
%\newcommand{\mw}{m_{ \sss W}}
%\newcommand{\mz}{m_{ \sss Z}}
%\newcommand{\mt}{m_{t}}



%%%%CMS_section.tex or whereabout
\newcommand{\SMWidth}{{\sc\small SMWidth}}
\newcommand{\FeynArts}{{\sc\small FeynArts}}
\newcommand{\HDecay}{{\sc\small HDecay}}
\newcommand{\OneLoop}{{\sc\small OneLoop}}
\newcommand{\MGaMC}{\aNLOs}
\newcommand{\MadLoop}{\ml}

%%%%%%%%%%%%%%% DENNER definitions %%%%%%%%%%%%

%% Abbreviations for environments
\def\beq{\begin{equation}}
\def\eeq{\end{equation}}
\def\beqar{\begin{eqnarray}}
\def\eeqar{\end{eqnarray}}
\def\barr#1{\begin{array}{#1}}
\def\earr{\end{array}}
\def\bfi{\begin{figure}}
\def\efi{\end{figure}}
\def\btab{\begin{table}}
\def\etab{\end{table}}
\def\bce{\begin{center}}
\def\ece{\end{center}}
\def\nn{\nonumber}
\def\nl{\nonumber\\}

%% shorthands for greek letters
\def\al{\alpha}
\def\be{\beta}
\def\ga{\gamma}
\def\de{\delta}
\def\la{\lambda}
\def\si{\sigma}
\def\Si{\Sigma}


%% new commands for cross referencing
\def\refeq#1{\mbox{(\ref{#1})}}
\def\reffi#1{\mbox{Figure~\ref{#1}}}
\def\reffis#1{\mbox{Figures~\ref{#1}}}
\def\refta#1{\mbox{Table~\ref{#1}}}
\def\reftas#1{\mbox{Tables~\ref{#1}}}
\def\refse#1{\mbox{Section~\ref{#1}}}
\def\refses#1{\mbox{Sections~\ref{#1}}}
\def\refapp#1{\mbox{App.~\ref{#1}}}
\def\citere#1{\mbox{Ref.~\cite{#1}}}
\def\citeres#1{\mbox{Refs.~\cite{#1}}}

%%physical units
\newcommand{\TeV}{\unskip\,\mathrm{TeV}}
\newcommand{\GeV}{\unskip\,\mathrm{GeV}}
\newcommand{\MeV}{\unskip\,\mathrm{MeV}}

%% roman symbols
\newcommand{\ri}{{\mathrm{i}}}
\newcommand{\rd}{{\mathrm{d}}}
\newcommand{\rS}{{\mathrm{S}}}
\newcommand{\rR}{{\mathrm{R}}}
\newcommand{\rT}{{\mathrm{T}}}
\newcommand{\rL}{{\mathrm{L}}}

%% calligraphic symbols
\renewcommand{\O}{{\cal O}}
\newcommand{\M}{{\cal{M}}}
\newcommand{\Mew}{{\tilde{\cal{M}}}}

% physical particles
\def\mathswitchr#1{#1}
\newcommand{\PW}{\mathswitchr W}
\newcommand{\PB}{\mathswitchr B}
\newcommand{\PZ}{\mathswitchr Z}
\newcommand{\PA}{\mathswitchr A}
\newcommand{\PH}{\mathswitchr H}
\newcommand{\Pf}{\mathswitch f}
\newcommand{\Pfbar}{\mathswitch \bar f}
\newcommand{\Pe}{\mathswitchr e}
\newcommand{\Pd}{\mathswitchr d}
\newcommand{\Pu}{\mathswitchr u}
\newcommand{\Ps}{\mathswitchr s}
\newcommand{\Pc}{\mathswitchr c}
\newcommand{\Pb}{\mathswitchr b}
\newcommand{\Pt}{\mathswitchr t}
\newcommand{\Pep}{\mathswitchr {e^+}}
\newcommand{\Pem}{\mathswitchr {e^-}}
\newcommand{\PWp}{\mathswitchr {W^+}}
\newcommand{\PWm}{\mathswitchr {W^-}}
\newcommand{\PWpm}{\mathswitchr {W^\pm}}

% particle masses
\def\mathswitch#1{\relax\ifmmode#1\else$#1$\fi}
\newcommand{\MW}{\mathswitch {M_\PW}}
\newcommand{\MZ}{\mathswitch {M_\PZ}}
\newcommand{\MH}{\mathswitch {M_\PH}}
\newcommand{\MHt}{\mathswitch {M_{\PH,\Pt}}}
\newcommand{\Me}{\mathswitch {m_\Pe}}
\newcommand{\Mt}{\mathswitch {m_\Pt}}
\newcommand{\Mb}{\mathswitch {m_\Pb}}
\newcommand{\GW}{\Gamma_{\PW}}
\newcommand{\GZ}{\Gamma_{\PZ}}

\newcommand{\ntad}{n_{\rm tad}}

% shorthands for SM parameters
\newcommand{\thw}{\mathswitch {\theta_\mathrm{w}}}
\newcommand{\cw}{\mathswitch {c_\mathrm{w}}}
\newcommand{\sw}{\mathswitch {s_\mathrm{w}}}
\newcommand{\GF}{\mathswitch {G_\mu}}
\newcommand{\NCf}{\mathswitch {N_{\mathrm{C}}^f}}
\newcommand{\NCt}{\mathswitch {N_{\mathrm{C}}^{\Pt}}}

% mathematical symbols
\newcommand{\lsim}
{\mathrel{\raisebox{-.3em}{$\stackrel{\displaystyle <}{\sim}$}}}
\newcommand{\gsim}
{\mathrel{\raisebox{-.3em}{$\stackrel{\displaystyle >}{\sim}$}}}
\newcommand{\Tr}{\mathop{\mathrm{Tr}}\nolimits}
\newcommand{\SU}{\mathrm{SU}}
\newcommand{\U}{\mathrm{U}}
\newcommand{\SUtwo}{\mathrm{SU(2)}}
\newcommand{\Uone}{\mathrm{U}(1)}

% various abbreviations
\def\ie{i.e.\ }
\def\eg{e.g.\ }
\def\cf{cf.\ }


\newcommand{\elm}{{\mathrm{em}}}
\newcommand{\ew}{{\mathrm{ew}}}
\newcommand{\sew}{{\mathrm{ew}}}
\newcommand{\htop}{{H,t}}
\newcommand{\weak}{{\mathrm{weak}}}
\newcommand{\real}{{\mathrm{real}}}
\newcommand{\virt}{{\mathrm{virt}}}
\newcommand{\soft}{{\mathrm{soft}}}
\newcommand{\coll}{{\mathrm{coll}}}
\newcommand{\rem}{{\mathrm{rem}}}
\newcommand{\symm}{{\mathrm{symm}}}
\newcommand{\asymm}{{\mathrm{asymm}}}
\newcommand{\fact}{{\mathrm{fact}}}
\newcommand{\nonfact}{{\mathrm{nfact}}}
\newcommand{\SC}{{\mathrm{LSC}}}
\renewcommand{\SS}{{\mathrm{SSC}}}
\newcommand{\cc}{{\mathrm{C}}}
\newcommand{\s}{{\mathrm{s}}}
\newcommand{\pre}{{\mathrm{PR}}}
\newcommand{\Yuk}{{\mathrm{Yuk}}}

%% commands for this paper
\newcommand{\eeWW}{{\Pe^+ \Pe^-\to \PW^+ \PW^-}}
\newcommand{\Wpff}{{\PW^+ \to f_1\bar f_2}}
\newcommand{\Wmff}{{\PW^- \to f_3\bar f_4}}
\newcommand{\eeWWffff}{\Pep\Pem\to\PW\PW\to 4f}
\newcommand{\eeffff}{\Pep\Pem\to 4f}
\newcommand{\eeffffg}{\eeffff\ga}
\newcommand{\bew}{b^{\ew}}
\newcommand{\besw}{\tilde{b}^{\ew}}

\newcommand{\cew}{C^{\ew}}
\newcommand{\csew}{\tilde{C}^{\ew}}
\newcommand{\dew}{D^{\ew}}
\newcommand{\dsew}{\tilde{D}^{\ew}}
\newcommand{\sNB}{\tilde{\NB}}
\newcommand{\NB}{N}
\newcommand{\GB}{V}

% shorthands for energy dependent single logarithms
\newcommand{\ls}{l(s)}
\newcommand{\lu}{l(\mu^2)}
\newcommand{\lQ}{l(Q^2)}
\newcommand{\lrM}{l(r_{kl},M^2)}
\newcommand{\lrMwithabs}{l(|r_{kl}|,M^2)}
\newcommand{\lrW}{l(r_{kl},\MW^2)}
\newcommand{\lrZ}{l(r_{kl},\MZ^2)}
\newcommand{\lsMa}{l(s,M_{\GB_a}^2)}
\newcommand{\lsM}{l(s,M^2)}
\newcommand{\lsW}{l(s,\MW^2)}
\newcommand{\lsf}{l(s,m_f^2)}
\newcommand{\lmuf}{l(\mu^2,m_f^2)}
\newcommand{\lmuZ}{l(\mu^2,\MZ^2)}
\newcommand{\lmuW}{l(\mu^2,\MW^2)}
\newcommand{\lsl}{l_{\cc}}
\newcommand{\lpr}{l_{\pre}}
\newcommand{\lYuk}{l_{\Yuk}}
\newcommand{\lZ}{l_{\PZ}}
% shorthands for constant single logarithms
\newcommand{\lWf}{l(\MW^2,m_f^2)}
\newcommand{\lWfnew}{l^{\rm reg}(\MW^2,m_f^2)}
\newcommand{\lWfsi}{l(\MW^2,m_{f_\si}^2)}
\newcommand{\lWk}{l(\MW^2,m_k^2)}
\newcommand{\lWla}{l(\MW^2,\la^2)}
\newcommand{\lWlanew}{l(\MW^2,Q^2)}
\newcommand{\lWlanewmu}{l(\MW^2,\mu^2)}

\newcommand{\lWfsii}{l(\MW^2,m_{f_{\si,i}}^2)}
\newcommand{\lWNB}{l(\MW^2,M_\NB^2)}
\newcommand{\lWa}{l(\MW^2,M_{\GB_a}^2)}
\newcommand{\lWZ}{l(\MW^2,\MZ^2)}
\newcommand{\lWM}{l(\MW^2,M^2)}
\newcommand{\ltW}{l(\Mt^2,\MW^2)}
\newcommand{\lHW}{l(\MH^2,\MW^2)}
\newcommand{\lemf}{l^\elm(m_f^2)}
\newcommand{\lemftau}{l^\elm(m_{f_\tau}^2)}
\newcommand{\lemfsi}{l^\elm(m_{f_\si}^2)}
\newcommand{\lemfsinew}{l^\elm(Q^2)}
\newcommand{\lem}{l^\elm(m^2)}
\newcommand{\lemk}{l^\elm(m_{k}^2)}
\newcommand{\lemphi}{l^\elm(m_{\varphi}^2)}
\newcommand{\leme}{l^\elm(m_\Pe^2)}
\newcommand{\lemW}{l^\elm(\MW^2)}
% shorthands for energy dependent double logarithms
\newcommand{\Ls}{L(s)}
\newcommand{\LrM}{L(|r_{kl}|,M^2)}
\newcommand{\LrMnoabs}{L(r_{kl},M^2)}
\newcommand{\Lrs}{L(|r_{kl}|,s)}
\newcommand{\LrMa}{L(|r_{kl}|,M_{\GB_a}^2)}
\newcommand{\LsM}{L(s,M^2)}
\newcommand{\LsW}{L(s,\MW^2)}
\newcommand{\LsZ}{L(s,\MZ^2)}
\newcommand{\Lsla}{L(s,\lambda^2)}
% shorthands for constant double logarithms
\newcommand{\Lkla}{L(m_k^2,\la^2)}
\newcommand{\Lklanew}{L^{\rm{reg}}(m_k^2,Q^2)}

\newcommand{\LWk}{L(\MW^2,m_k^2)}
\newcommand{\LWf}{L(\MW^2,m_f^2)}
\newcommand{\LZW}{L(\MZ^2,\MW^2)}
\newcommand{\LWla}{L(\MW^2,\lambda^2)}
\newcommand{\LWlanew}{L(\MW^2,Q^2)}

% shorthands for electromagnetic double logarithms
\newcommand{\Lemk}{L^\elm(s,\lambda^2,m_k^2)}
\newcommand{\Lemknew}{L^\elm(s,Q^2,m_k^2)}

\newcommand{\Lemphi}{L^\elm(s,\lambda^2,m_\varphi^2)}
\newcommand{\Lemf}{L^\elm(s,\lambda^2,m_f^2)}
\newcommand{\Lemftau}{L^\elm(s,\lambda^2,m_{f_\tau}^2)}
\newcommand{\Leme}{L^\elm(s,\lambda^2,m_e^2)}
\newcommand{\LemW}{L^\elm(s,\lambda^2,\MW^2)}
\newcommand{\lrs}{\log{\frac{|r_{kl}|}{s}}}
\newcommand{\lrsalpha}{l(|r_{kl}|,s)}

\newcommand{\ltu}{\log{\frac{t}{u}}}
\newcommand{\lts}{\log{\frac{|t|}{s}}}
\newcommand{\lus}{\log{\frac{|u|}{s}}}
\newcommand{\lsu}{\log{\frac{s}{|u|}}}

\newcommand{\TO}{\rightarrow}
\newcommand{\mglong}{{\sc\small Mad\-Graph5\_aMC\-@NLO}}
\newcommand{\HELAS}{{\sc\small Helas}}
\newcommand{\ALOHA}{{\sc\small Aloha}}
\newcommand{\mgshort}{{\sc\small MG5\_aMC}}
\newcommand{\denpoz}{{\sc\small DP}}

\newcommand{\deltaEW}{\delta^{\rm EW}_{\rm LA}}
\newcommand{\deltaQCD}{\delta^{\rm QCD}_{\rm LA}}
\newcommand{\Ltop}{L^t(s)}
\def\Las#1{l^{\as}(#1)}
\newcommand{\dmtQCD}{(\de\Mt)^{\rm QCD}}
\newcommand{\MSbar}{{\rm \overline{MS}}}

\newcommand\clH{{\mathbb H}}
\newcommand\clS{{\mathbb S}}


\newcommand\yij{y_{ij}}
\newcommand\procB{r_{\sss B}}
\newcommand\procR{r_{\sss R}}
\newcommand\allproc{{\cal R}}
\newcommand\allprocnpo{\allproc_{n+1}}
\newcommand\allprocn{\allproc_{n}}

%%%%rfsf.tex
\newcommand\Kern{{\cal K}}
\newcommand\KernMC{{\cal K}_{\sss\rm MC}}
\newcommand\kinHi{\Xi_{\clH,i}}
\newcommand\kinSi{\Xi_{\clS,i}}
\newcommand\kinHz{\Xi_{\clH,0}}
\newcommand\kinSz{\Xi_{\clS,0}}
\newcommand\kinHo{\Xi_{\clH,1}}
\newcommand\kinSo{\Xi_{\clS,1}}
\newcommand\kinHt{\Xi_{\clH,2}}
\newcommand\kinSt{\Xi_{\clS,2}}
\newcommand\kinHN{\Xi_{\clH,N}}
\newcommand\kinSN{\Xi_{\clS,N}}
\newcommand\muSimn{\mu_{i,{\min}}^{(\clS)}}
\newcommand\muSimx{\mu_{i,{\max}}^{(\clS)}}
\newcommand\muHimn{\mu_{i,{\min}}^{(\clH)}}
\newcommand\muHimx{\mu_{i,{\max}}^{(\clH)}}
\newcommand\muSNmn{\mu_{N,{\min}}^{(\clS)}}
\newcommand\muSNmx{\mu_{N,{\max}}^{(\clS)}}
\newcommand\muHNmn{\mu_{N,{\min}}^{(\clH)}}
\newcommand\muHNmx{\mu_{N,{\max}}^{(\clH)}}

%%%%fourlep.tex
\newcommand\Wa{W^{(\alpha)}}
\newcommand\conf{{\cal K}}
\newcommand\confn{\conf_n}
\newcommand\confnpo{\conf_{n+1}}
\newcommand\confnpoa{\conf_{n+1}^{(\alpha)}}
\newcommand\confnpoE{\conf_{n+1}^{(E)}}
\newcommand\confnpoC{\conf_{n+1}^{(C)}}
\newcommand\confnpoS{\conf_{n+1}^{(S)}}
\newcommand\confnpoSC{\conf_{n+1}^{(SC)}}
\newcommand\confH{\conf^{(\clH)}}
\newcommand\confS{\conf^{(\clS)}}
\newcommand\confHn{\conf_n^{(\clH)}}
\newcommand\confSn{\conf_n^{(\clS)}}
\newcommand\confHi{\conf_i^{(\clH)}}
\newcommand\confSi{\conf_i^{(\clS)}}
\newcommand\confHN{\conf_N^{(\clH)}}
\newcommand\confSN{\conf_N^{(\clS)}}
\newcommand\confHip{\conf_{p+i}^{(\clH)}}
\newcommand\confSip{\conf_{p+i}^{(\clS)}}
\newcommand\confHNp{\conf_{p+N}^{(\clH)}}
\newcommand\confSNp{\conf_{p+N}^{(\clS)}}
\newcommand\sigmaLO{\sigma^{\sss\rm (LO)}}
\newcommand\sigmaNLO{\sigma^{\sss\rm (NLO)}}
\newcommand\sigmaNLOa{\sigma^{{\sss (\rm NLO},\alpha{\sss )}}}
\newcommand\sigmaNLOE{\sigma^{\sss {(\rm NLO},E)}}
\newcommand\sigmaNLOS{\sigma^{\sss {(\rm NLO},S)}}
\newcommand\sigmaH{\sigma^{\sss (\clH)}}
\newcommand\sigmaS{\sigma^{\sss (\clS)}}
\newcommand\sigmaHi{\sigma_i^{\sss (\clH)}}
\newcommand\sigmaSi{\sigma_i^{\sss (\clS)}}
\newcommand\sigmaHN{\sigma_N^{\sss (\clH)}}
\newcommand\sigmaSN{\sigma_N^{\sss (\clS)}}
\newcommand\bsigmaHi{\bar{\sigma}_i^{\sss (\clH)}}
\newcommand\bsigmaSi{\bar{\sigma}_i^{\sss (\clS)}}
\newcommand\bsigmaHN{\bar{\sigma}_N^{\sss (\clH)}}
\newcommand\bsigmaSN{\bar{\sigma}_N^{\sss (\clS)}}
\newcommand\bsigmaHn{\bar{\sigma}_n^{\sss (\clH)}}
\newcommand\bsigmaSn{\bar{\sigma}_n^{\sss (\clS)}}
\newcommand\bsigmaHip{\bar{\sigma}_{p+i}^{\sss (\clH)}}
\newcommand\bsigmaSip{\bar{\sigma}_{p+i}^{\sss (\clS)}}
\newcommand\bsigmaHNp{\bar{\sigma}_{p+N}^{\sss (\clH)}}
\newcommand\bsigmaSNp{\bar{\sigma}_{p+N}^{\sss (\clS)}}
\newcommand\hsigmaHip{\hat{\sigma}_{p+i}^{\sss (\clH)}}
\newcommand\hsigmaSip{\hat{\sigma}_{p+i}^{\sss (\clS)}}
\newcommand\sigmaMC{\sigma^{\sss\rm (MC)}}
\newcommand\sigmaMCc{\sigma^{{\sss (\rm MC},c{\sss )}}}
\newcommand\sigmaMCz{\sigma^{{\sss (\rm MC},0{\sss )}}}
\newcommand\sigmaMCD{\sigma^{{\sss (\rm MC},{\sss D)}}}
\newcommand\hsigmaMC{\hat{\sigma}^{\sss\rm (MC)}}
\newcommand{\meas}{\chi}
\newcommand\Gfun{{\cal G}}



\newcommand\HW{{\sc\small Herwig}}

\newcommand\PY{{\sc\small Pythia}}
\newcommand\PYs{{\sc\small Pythia6}}

\newcommand\lum{{\cal L}}
\newcommand\lumMC{\lum^{\sss\rm (MC)}}

\newcommand\showxi{\xi^{(l)}}
\newcommand\showz{z^{(l)}}
\newcommand\showxiij{\showxi_{ij}}
\newcommand\showzij{\showz_{ij}}
\newcommand\stepfMC{\Theta^{\sss\rm (MC)}}

\newcommand\mcatnlo{{\sc\small MC@NLO}}

\newcommand\GenMC{{\cal F}_{\mbox{\tiny MC}}}
\newcommand\GenMCk{{\cal F}_{\mbox{\tiny MC}}^{(k)}}
\newcommand\GenNLO{{\cal F}_{\mbox{\tiny MC@NLO}}}
\newcommand\GenNLOk{{\cal F}_{\mbox{\tiny MC@NLO}}^{(k)}}
\newcommand\GenFxFx{{\cal F}_{\mbox{\tiny FxFx}}}
\newcommand\GenFxFxi{{\cal F}_{\mbox{\tiny FxFx}}^{(i)}}

\newcommand\MadFKS{{\sc\small MadFKS}}
\newcommand\amcatnlo{{\sc\small aMC@NLO}}

\newcommand\bu{\bar{u}}
\newcommand\bd{\bar{d}}



\newcommand\qpart{q^\star}



%%%%ffms.tex

\newcommand\tphsp{d\widetilde{\phi}}
\newcommand\tphspn{\tphsp_{n}}
\newcommand\tphspnij{\tphsp_{n}^{ij}}
\newcommand\Phsp{\Phi}
\newcommand\Phspn{\Phsp^{(n)}}
\newcommand\Phspnpo{\Phsp^{(n+1)}}
\newcommand\isubrmv{\remove{i}{0.125}}
\newcommand\jsubrmv{\remove{j}{0.145}}

\newcommand\luma{\lum^{(\alpha)}}
\newcommand\FKSpairsred{\overline{{\cal P}}_{\sss\rm FKS}}
\newcommand\symmnpoij{\symm_{ij}^{(n+1)}}

\newcommand{\muS}{\mu_S}



 


%%%%%%%%%%%%%%% DENNER definitions end %%%%%%%%%%


\newcommand{\MzSM}{\mathcal{M}_0^{\rm SM}}
\newcommand{\MzSMp}{\mathcal{M}_{0'}^{\rm SM}}
\newcommand{\MoSM}{\mathcal{M}_{1, {\rm EW}}^{\rm SM}}
\newcommand{\dMSM}{\delta \mathcal{M}^{\rm SM}_{{\rm EW}}}
\newcommand{\MzNP}{\mathcal{M}_0^{\rm NP}}
\newcommand{\MzNPp}{\mathcal{M}_{0'}^{\rm NP}}
\newcommand{\MoNP}{\mathcal{M}_{1, {\rm EW}}^{\rm NP}}
\newcommand{\dMNP}{\delta \mathcal{M}^{\rm NP}}
\newcommand{\SM}{\rm SM}
\newcommand{\NP}{\rm NP}

\newcommand{\MoSMQCD}{\mathcal{M}_{1, {\rm QCD}}^{\rm SM}}
\newcommand{\MoNPQCD}{\mathcal{M}_{1, {\rm QCD}}^{\rm NP}}


\newcommand{\proptoinverse}{\mathrel{\mskip1mu\reflectbox{$\propto$}\mskip-1mu}}
\newcommand{\LONP}{\LO^{\NP}}
\newcommand{\LONPtwo}{\LO^{\NP^2}}
\newcommand{\NLOQCDNP}{\NLO^{\NP}_{\rm QCD}}
\newcommand{\NLOQCDNPtwo}{\NLO^{\NP^2}_{\rm QCD}}
\newcommand{\NLOQCDEWSLNP}{\NLO^{\NP}_{\rm QCD + EWSL}}
\newcommand{\NLOQCDEWSLNPtwo}{\NLO^{\NP^2}_{\rm QCD+ EWSL}}



\newcommand{\SDK}{\rm SDK}
\newcommand{\SDKw}{{\SDK}_{\rm weak}}
\newcommand{\SDKz}{{\SDK}_{0}}

\newcommand{\sigmaSM}{\sigma^{\rm SM}}
\newcommand{\sigmaINT}{\sigma^{\rm INT}}
\newcommand{\sigmaSQ}{\sigma^{\rm SQ}}




%%%%%%%%%%%%%%

%\newcommand{\CS}[1]{{\textcolor{magenta}{[CS: #1]}}}
%\newcommand{\HF}[1]{{\textcolor{blue}{[HF: #1]}}}
%\newcommand{\EV}[1]{{\textcolor{teal}{[EV: #1]}}}
%\newcommand{\KM}[1]{{\textcolor{violet}{[KM: #1]}}}
%\newcommand{\MA}[1]{{\textcolor{orange}{[MZ: #1]}}}

\newcommand{\textDP}[1]{{\textcolor{violet}{ #1}}}

\maxdeadcycles=2000
\extrafloats{100}

\title{The story with less, but still many, plots}

\author[a]{John}
\author[a]{Paul}
\author[a]{George}
\author[a]{Ringo}

\affiliation[a]{Abbey Rd.~London, Regno Unito
}


\preprint{
\begin{flushright}
%TIF-UNIMI-2024-19
%{x2}
%{x3}
\end{flushright}
}

\abstract{
Showing that myths are false.  All the plots are improvable, both in layout and precision. Many more plots are available in the other document. For luminosities and PDFs I consider only the 10 TeV case here: 1, 3 and 30 TeV are in the other. 
\\
{Please pay attention that in some plots (Longitudinal=0, Right=R, Left=L) in some others (Longitudinal=L, Right=+ or p, Left=- or m) }
}

\keywords{NLO Computations, Monte Carlo, Collider Physics, BDSM}

\begin{document}
\maketitle
\flushbottom


\section{Intro}

A VBF process at muon collider can either be described via a Matrix-Element (ME) approach or exploiting the Effective-Vector-boson-Approximation. 
The former exactly takes into account all the contributions at a given perturbative order for the corresponding process, including those   contributions originating from  topologies that have nothing in common with VBF.  The latter leads to a much easier computation, which is the reason why it is appealing, since it features two legs less in the final state. However, topologies different than VBS are completely neglected and also for them 
power corrections ($(\mv/\sqrt{s})^n$ with $n\ge 0$) are not under control. \footnote{In all this note we neglect the PDF or the ``EVA'' of the muon in the muon for such comparison. In the case of $2\to 4$ processes considered we simply consider the ME so $\hat s= s$.}, and logarithm-enhanced terms depend on a factorisation scale, which is as usual arbitrary. Such logarithms are precisely the origin of the common lore: EVA is a good description of the process and leads to a large contribution, so VBF is the dominant process and we can use EVA instead of VBF.

First of all we want to explore at what level the previous statement is true. In doing so we compare  ME vs.~EVA and for the EVA we use two different parameterisations. Using the notation of Ref.~\cite{Garosi:2023bvq}, the EVA can be defined for the  transverse and longitudinal $W^-$ boson as 
\begin{eqnarray}
   { \color{blue} f_{W^-_\pm}^{(\alpha)}(x, Q^2)} &=& \int_{m_\mu^2}^{Q^2} d p_T^2 \frac{1}{2} \frac{d \mathcal{P}_{\psi \to W_T \psi}}{d p_T^2}(x, p_T^2) \, 
    = \int_{m_\mu^2}^{Q^2} d p_T^2 \, \frac{\alpha_2}{8 \pi} \frac{p_T^2}{(p_T^2 + (1-x) m_W^2)^2} P_{V_\pm f_L}^f(x) = \nonumber \\
    &=& { \color{blue}\frac{\alpha_2}{8 \pi} P_{V_\pm f_L}^f(x) \left( \log \frac{Q^2 + (1-x) m_W^2}{m_{\mu}^2 + (1-x) m_W^2} 
    - \frac{Q^2}{Q^2 + (1-x) m_W^2} \right) }= \nonumber \\
    &\approx& \frac{\alpha_2}{8 \pi} P_{V_\pm f_L}^f(x) \left( \log \frac{Q^2}{m_W^2} - \log(1-x) - 1\right) + \mathcal{O}\left( \frac{m_W^2}{Q^2} \right)~, \label{eq:EWA_WT} \\
%
    {\color{blue} f_{W^-_L}^{(\alpha)}(x, Q^2)} &=& \int_{0}^{Q^2} d p_T^2 \frac{1}{2} \frac{d \mathcal{P}_{\psi \to W_L \psi}}{d p_T^2} (x, p_T^2) \, 
    = \int_{0}^{Q^2} d p_T^2 \, \frac{\alpha_2}{4 \pi} \frac{m_W^2}{(p_T^2 + (1-x) m_W^2)^2} \frac{(1-x)^2}{x} = \nonumber \\
    &=& { \color{blue} \frac{\alpha_2}{4 \pi} \frac{1-x}{x}
    \frac{Q^2}{Q^2 + (1-x) m_W^2}}
    \approx \frac{\alpha_2}{4 \pi} \frac{1-x}{x} + \mathcal{O}\left( \frac{m_W^2}{Q^2} \right) ~,\label{eq:EWA_WL}
\end{eqnarray}
and analogously for the $Z$ and $Z/\gamma$ PDFs
\begin{eqnarray}
    {\color{blue}f_{Z_\pm}^{(\alpha)}(x, Q^2)} &=& 
    {\color{blue}\frac{\alpha_2}{4 \pi c_W^2} \left(P_{V_\pm f_L}^f(x) (Q^Z_{\mu_L})^2 + P_{V_\pm f_R}^f(x) (Q^Z_{\mu_R})^2 \right) }\nonumber \\
    && \quad {\color{blue}\left( \log \frac{Q^2 + (1-x) m_Z^2}{m_{\mu}^2 + (1-x) m_Z^2} 
    - \frac{Q^2}{Q^2 + (1-x) m_Z^2} \right)} ~, \label{eq:EWA_ZT} \\
%
    f_{Z/\gamma_\pm}^{(\alpha)}(x, Q^2) &=& 
    -\frac{\sqrt{\alpha_\gamma \alpha_2}}{2 \pi c_W} \left(P_{V_\pm f_L}^f(x) Q^Z_{\mu_L} + P_{V_\pm f_R}^f(x) Q^Z_{\mu_R} \right) \log \frac{Q^2 + (1-x) m_Z^2}{m_{\mu}^2 + (1-x) m_Z^2}  ~, \label{eq:EWA_ZgaT} \\
%
   {\color{blue} f_{Z_L}^{(\alpha)}(x, Q^2)} &=&
    {\color{blue}\frac{\alpha_2}{2 \pi c_W^2} \frac{1-x}{x} \left( (Q^Z_{\mu_L})^2 + (Q^Z_{\mu_R})^2\right) 
    \frac{Q^2}{Q^2 + (1-x) m_Z^2} }~,\label{eq:EWA_ZL}
\end{eqnarray}
where $P_{V_+ f_L}^f(x) = P_{V_- f_R}^f(x) = (1-x)^2 / x$ and $P_{V_- f_L}^f(x) = P_{V_+ f_R}^f(x) = 1 / x$.
The muon mass here serves as an IR cutoff for the logarithm in the transverse case to cure the $x\to 1$ limit, while it is neglected it in the other terms. Notably, the $W^+$ has no contribution at this order.
In our comparison with ME we use the formula in {\color{blue} blue}, therefore we are not considering the case of the mixed photon-$Z$ and of the photon itself.
In other words, just $W$ and $Z$ bosons.

The other EVA parameterisation  according to Ref.~\cite{Ruiz:2021tdt}, assuming $q^2$ evolution,
 for polarized $V_\lambda\in\{W_\lambda^\pm,Z_\lambda\}$ from LH $(\tilde{f}_L)$ and RH $(\tilde{f}_R)$ fermions in the hard scattering frame are 
\begin{subequations}
\label{eq:formalism_pdfs_q2}
\begin{align}
\tilde{f}_{V_+/f_L}(\xi,\mu_f^2) 	&= \frac{g_V^2}{4\pi^2} \frac{g_L^2(1-\xi)^2}{2\xi} \log \left[\frac{\mu_f^2}{M_V^2}\right], \\
\tilde{f}_{V_-/f_L}(\xi,\mu_f^2) 	&= \frac{g_V^2}{4\pi^2} \frac{g_L^2}{2\xi} \log \left[\frac{\mu_f^2}{M_V^2}\right], \\
\tilde{f}_{V_0 / f_L}(\xi,\mu_f^2) 	&= \frac{g_V^2}{4\pi^2} \frac{g_L^2(1-\xi)}{\xi},\\  
\tilde{f}_{V_+/f_R}(\xi,\mu_f^2) 	&= \left(\frac{g_R}{g_L}\right)^2 \times \tilde{f}_{V_-/f_L}(\xi,\mu_f^2), \\
\tilde{f}_{V_-/f_R}(\xi,\mu_f^2) 	&= \left(\frac{g_R}{g_L}\right)^2 \times \tilde{f}_{V_+/f_L}(\xi,\mu_f^2), \\
\tilde{f}_{V_0/f_R}(\xi,\mu_f^2) 	&= \left(\frac{g_R}{g_L}\right)^2 \times \tilde{f}_{V_0/f_L}(\xi,\mu_f^2). 
\end{align}
\end{subequations}
%
with couplings defined in Table~\ref{tab:ewa_coup}.

Here we report on purpose the two definitions using the two different notations in order to check their consistency.
First of all we note that in Eq.~\eqref{eq:formalism_pdfs_q2}, the quantity $g_V$ is wrong and should be replaced with $\tilde{g}$. Second,
polarized EVA for \textit{unpolarized} muon beams, denoted by $\tilde{f}_{V_\lambda/\mu^\pm}$, can be constructed from those PDFs for \textit{polarized} muons, denote by $\tilde{f}_{V_\lambda/\mu^\pm_\lambda}$, through the relation
\begin{equation}
 \tilde{f}_{V_\lambda/\mu^\pm}(\xi,\mu_f) =
 \cfrac{\tilde{f}_{V_\lambda/\mu^\pm_L}(\xi,\mu_f) + \tilde{f}_{V_\lambda/\mu^\pm_R}(\xi,\mu_f)}{2}
\label{eq:pdfDef_unpolarizedMuon}
\end{equation}
%
and instead 
%
\begin{equation}
 \tilde{f}_{V_T/\mu^\pm}(\xi,\mu_f) =
\tilde{f}_{V_+/\mu^\pm}(\xi,\mu_f) + \tilde{f}_{V_-/\mu^\pm}(\xi,\mu_f).
\label{eq:pdfDef_unpolarizedMuon}
\end{equation}



\begin{table}[!t]
\begin{center}
\resizebox{\textwidth}{!}{
\renewcommand*{\arraystretch}{0.95}
\begin{tabular}{c|c|c|c|c|c}
\hline\hline
\multirow{2}{*}{Vertex} & Coupling  & \multirow{2}{*}{$g_R^f$}  & \multirow{2}{*}{$g_L^f$}  & \multirow{2}{*}{$g_V^f$}  & \multirow{2}{*}{$g_A^f$} \\
	& strength	&	& &  & \\
\hline\hline
$V-f-f'$	& $\tilde{g}$				& $(g_V^f+g_A^f)$		& $(g_V^f-g_A^f)$		& $\frac{(g_R+g_L)}{2}$ 	& $\frac{(g_R-g_L)}{2}$ \\
$\gamma-f-f$	& $e Q^f$				& $1$				& $1$				& $1$				& $0$\\
$Z-f-f$	    & $\frac{g}{\cos\theta_W}$	& $-Q^f\sin^2\theta_W$	& $(T_3^f)_L-Q^f\sin^2\theta_W$				& $\frac{1}{2}(T_3^f)_L-Q^f\sin^2\theta_W$ 			& $-\frac{1}{2}(T_3^f)_L$ \\
$W-f-f'$	& $\frac{g}{\sqrt{2}}$			& $0$				& $1$				& $\frac{1}{2}$ 			& $-\frac{1}{2}$ \\
\hline\hline
\end{tabular}
}
\caption{
EW chiral couplings and coupling strength normalizations used in the EVA for fermions $f,f'$ with weak isospin charge $(T_3^f)_L=\pm1/2$ and electric charge $Q^f$, with normalization $Q^\ell=-1$.
}
\label{tab:ewa_coup}
\end{center}
\end{table}

\pnote{TEXT TO BE WRITTEN FOR THE CASE OF THE ANTIMUON}


After having compared ME vs EVA predictions for some representative distribution and cuts set-ups, which are mostly the same considered in the pheno paper, we will discuss then the second step: moving from the EVA to the PDFs, and we use the {\small \sc LePDF} parameterisation.
We will compare individual PDFs and luminosities against their EVA counterparts.

For reference purpose we put here the PDFs at 10 TeV for the $W$, $Z$ and photon in {\small \sc LePDF}, T and L are transverse and longitudinal here.

\begin{figure}[!t]
\includegraphics[width=0.46\linewidth]{Notebooks/PlotPDFs/alltogether/10TeV_mu-scaleQ.pdf}
\includegraphics[width=0.46\linewidth]{Notebooks/PlotPDFs/alltogether/10TeV_mu-scaleQsqrtx.pdf}
\includegraphics[width=0.46\linewidth]{Notebooks/PlotPDFs/alltogether/10TeV_mu-scaleQx.pdf}
\caption{PDF of the $\mu^-$ with different scale choices. \label{fig:allmum}}
\end{figure}


\begin{figure}[!t]
\includegraphics[width=0.46\linewidth]{Notebooks/PlotPDFs/alltogether/10TeV_mu+scaleQ.pdf}
\includegraphics[width=0.46\linewidth]{Notebooks/PlotPDFs/alltogether/10TeV_mu+scaleQsqrtx.pdf}
\includegraphics[width=0.46\linewidth]{Notebooks/PlotPDFs/alltogether/10TeV_mu+scaleQx.pdf}
\caption{PDF of the $\mu^{\color{red} +}$ with different scale choices, for the same partons. \label{fig:allmup}}
\end{figure}


We also show representative luminosities using the same PDFs, where the luminosity for producing a system of mass $M$ in a collider with squared energy $s$ (so $\tau=M^2/s$) is defined as

\begin{equation}
\mathcal L(M,s,Q)=\int_{M^2/s}^1 \frac{1}{x} f_1(x,Q) f_2(M^2/(x s),Q) dx\, ,
\end{equation}
where $Q$ is the factorisation scale chosen.

A physically motivated choice for $Q$ is precisely $M$ or a variation by a factor of 2 around it. In fact, as we will see later the case $M/2$ seems to better reproduce results from ME. Nevertheless, plugging $Q=M$ in the definition before we recover the more common

\begin{equation}
\mathcal L(M,s,Q=M)=\int_{\tau}^1 \frac{1}{x} f_1(x,M) f_2(\tau/x,M) dx\, , \label{eq:lumicommon}
\end{equation}

The three choices of scale in Figs.~\ref{fig:allmum} and \ref{fig:allmup} correspond therefore to 

\begin{itemize}
\item $Q=\sqrt{s}$, which is not very well suited for the small-$x$ region where PDFs of $Z$ and $W$ are relevant. If $Q=M$ as in Eq.~\eqref{eq:lumicommon} this case is equivalent only for $x\to 1$.
\item$Q=\sqrt{x} \sqrt{s}$: If $Q=M$ as in Eq.~\eqref{eq:lumicommon} this case is equivalent in the region $x\to \sqrt{\tau}$. For luminosities as $WW$ or $ZZ$ this is the best case to look at, in my opinion. Take $M$ and $S$, look at $\sqrt{\tau}$ and one can have a comparison for the bulk of the contribution of the different PDFs
\item $Q=x \sqrt{s}$: If $Q=M$ as in Eq.~\eqref{eq:lumicommon} this case is equivalent in the region $x \to \tau$. It is the opposite of the region $x\to 1$ and indeed, for $M=\mw$, where EW PDFs collapses, $\tau \simeq 0.008$, which we see in the plot.    
\end{itemize}


In Fig.~\ref{fig:lumicomparison} we show the comparison of representative luminosities with {\small \sc LePDF}, with the scale set at $Q=M/2$.
Comparisons for same luminosities with different polarisation are shown in the corresponding section later.


%NIntegrate[ 1/x \[Mu]EVApartv2[part1, m\[Mu]\[Mu]*muoverm\[Mu]\[Mu], 
%   x] \[Mu]bEVApartv2[part2, m\[Mu]\[Mu]*muoverm\[Mu]\[Mu], 
%   m\[Mu]\[Mu]^2/(x sqrts0^2)], {x, m\[Mu]\[Mu]^2/sqrts0^2, 1}  

\begin{figure}
\includegraphics[width=0.9\linewidth]{Notebooks/PlotLumi/10TeV/lumis/plotgammaWZ.pdf} 
\caption{Comparison of luminosities \label{fig:lumicomparison}}
\end{figure}

 
\clearpage

\section{$ZZ$ initiated $t \bar t$ production}

\begin{figure}[!t]
\includegraphics[width=0.46\linewidth]{Notebooks/PlotDistr/ZZ_tt/10TeVcuts/plotmtt.pdf}
\includegraphics[width=0.46\linewidth]{Notebooks/PlotDistr/ZZ_tt/10TeVcuts/plotmttratio1.pdf}
\includegraphics[width=0.46\linewidth]{Notebooks/PlotDistr/ZZ_tt/10TeVcuts/plotmttratio2.pdf}
\caption{With cuts, $m(t \bar t)$ distributions. \label{fig:cutsZZtt}}
\end{figure}


\begin{figure}[!t]
\includegraphics[width=0.46\linewidth]{Notebooks/PlotDistr/ZZ_tt/10TeVcuts/plotetat.pdf}
\includegraphics[width=0.46\linewidth]{Notebooks/PlotDistr/ZZ_tt/10TeVcuts/plotetatratio1.pdf}
\includegraphics[width=0.46\linewidth]{Notebooks/PlotDistr/ZZ_tt/10TeVcuts/plotetatratio2.pdf}
\caption{With cuts, $\eta(t)$ distributions. \label{fig:cutsZZtt2}}
\end{figure}



We start analysing the $ZZ\to t \bar t$ process, which is a VBF subcomponent of the complete process (ME) $\mu^+\mu^-\to\mu^+\mu^- t \bar t$. As it is clear, this process involves also the VBF process $\gamma \gamma \to t \bar t$, which is dominant. This can be guessed from Fig.~\ref{fig:lumicomparison} and by the fact that $\log(m(t\bar t)/m_\mu)\gg \log(m(t\bar t)/m_Z)$. Moreover, this process features also a $\mu^+\mu^-\to- t \bar t \gamma (\gamma \to\mu^+\mu^-)$ topology, which in a massless computation would be removed at NNLO. \footnote{Still, we have verified that at 10 TeV this contribution to be negligible w.r.t.~our NNLO total phenomenological prediction when a cut on $m(\mu^+ \mu-)$ is applied }Luckily, for such process the purely QED and purely weak component of the $\mu^+\mu^-\to\mu^+\mu^- t \bar t$ amplitude can be separated in a gauge invariant way, therefore we veto the photons in the amplitude. In order to be as close as possible to the VBF configuration we avoid also the process $\mu^+\mu^-\to t \bar t Z (Z \to\mu^+\mu^- )$, by actually generating the ME as $\mu^+e^-\to\mu^+e^- t \bar t$ and setting the mass and the Yukawa of the electron equal to the one of the muon. We denote in the plots such prediction as ME no-QED.

We start with a set of cuts defined as follows
%
\begin{equation}
m(t \bar t) > 500~\gev,~~ p_T(t/ \bar t) > 150~\gev,~~ |\eta(t/\bar t)|<2.5 
\end{equation}
%
we show in Fig.~\ref{fig:cutsZZtt} the $m(t \bar t)$ distribution at ME-noQED versus the EVA predictions using the notation of Ref.~\cite{Garosi:2023bvq} with a central scale definition for the EVA $\mu=m(t \bar t) /2$ and a variation around it by a factor of 2.
We also explicitly  show the ratio of the EVA predictions with the ME-noQED one. The same plot is also shown for parameterisation of the EVA from Ref.~\cite{Ruiz:2021tdt}, which we denoted in the plots as ``EVA only Log[mu/MV]''.

For both EVA definitions we observe a good agreement within the scale uncertainty band with the EVA-noQED. The case $\mu=m(t \bar t)/2$ is actually at the a few percents from the  EVA-noQED, and at 10\% for the case  ``EVA only Log[mu/MV]''. Still the scale uncertainty band is at the $\pm40\%$ level.

This good agreement deteriorates if the $\eta(t)$ distribution is considered, see Fig.~\ref{fig:cutsZZtt2} . \pnote{I still found it surprising and I wonder if there is a bug, like that EVA results could be in the partonic res frame. The same would apply in all the other eta plots. Such a bug for sure for the invariant mass because it is Lorentz-invariant. To be checked!} 
The agreement deteriorates also if we remove the cuts and we consider the full-inclusive levels. Corresponding plots are present in Figs.~\ref{fig:nocutsZZtt} and \ref{fig:nocutsZZtt2}. 

It is worth to remind the reader that the process we are discussing is a very small fraction of the entire $\mu^+\mu^-\to\mu^+\mu^- t \bar t$ process, where photon-initiated contributions dominate, expecially at small $m(t \bar t)$. Nevertheless it is interesting to check if the contribution of the PDFs may be relevant here, {\it i.e.} if the difference w.r.t.~the EVA predictions are larger than their uncertainty and their difference w.r.t the ME-noQED. For the case of $ZZ\to t \bar t$ we have just seen that uncertainty are larger and differences with   ME-noQED are vey small, for $\mu=m(t \bar t) /2$ and not using the ``EVA only Log[mu/MV]''.



 




\begin{figure}[!t]
\includegraphics[width=0.46\linewidth]{Notebooks/PlotDistr/ZZ_tt/10TeVnocuts/plotmtt.pdf}
\includegraphics[width=0.46\linewidth]{Notebooks/PlotDistr/ZZ_tt/10TeVnocuts/plotmttratio1.pdf}
\includegraphics[width=0.46\linewidth]{Notebooks/PlotDistr/ZZ_tt/10TeVnocuts/plotmttratio2.pdf}
\caption{No cuts, $m(t \bar t)$ distributions. \label{fig:nocutsZZtt}}
\end{figure}



\begin{figure}[!t]
\includegraphics[width=0.46\linewidth]{Notebooks/PlotDistr/ZZ_tt/10TeVnocuts/plotetat.pdf}
\includegraphics[width=0.46\linewidth]{Notebooks/PlotDistr/ZZ_tt/10TeVnocuts/plotetatratio1.pdf}
\includegraphics[width=0.46\linewidth]{Notebooks/PlotDistr/ZZ_tt/10TeVnocuts/plotetatratio2.pdf}
\caption{No cuts, $\eta(t)$ distributions. \label{fig:nocutsZZtt2}}
\end{figure}




We start by showing in Fig.~\ref{fig:lumiZZpol}, the luminosities for different combinations of polarisations for the $Z$ bosons, both using the PDFs (solid) and the EVA (dashed). For some of them also we show in Figs.~\ref{fig:lumiratioZZpol} and \ref{fig:lumiratioZZpol2} the ratio of PDFs (black), EVA (red) and ``EVA only Log[mu/MV]'' over the PDFs prediction at the scale $\mu=m(t \bar t) /2$. We do it also taking only in the numerator of the ratio  $\mu=m(t \bar t) /2 \time 2$ (dashed) and $\mu=m(t \bar t) /2 / 2$ (dot-dashed). In other words, when the solid-red line is outside the band between the black-dashed and black-dot-dashed lines it means that the difference between the EVA and the PDFs at the central scale is larger than the scale uncertainties of the PDFs.

As can be seen, the situation is completely different when transverse $Z$, both left handed and right handed  are considered w.r.t. the case of longitudinal $Z$. While at large invariant masses the longitudinal boson are negligible, at the threshold and still in the region where EVA and ME-noQED agrees they are the dominant contribution, we have verified it explicitly for the cut employed here. For the case of the individual PDFs rather than luminosities this can be better seen directly in Appendix \ref{app:PDFsEVA_Z}.


\pnote{I am not sure what conclude here. Seems like PDFs effects, i.e.~resummation is important. But I do not understand why $Z_L$ increase so much. Looking at the LePDF paper they convey the opposite message, but in reality they do this comparisons looking at PDFs and setting $Q=\sqrt{s}$ and in that case the effects on the transverse Z or W are larger than for longitudinal. One can see it also in Appendix \ref{app:PDFsEVA_Z}}



 




%\begin{figure}[!t]
%\includegraphics[width=0.46\linewidth]{Notebooks/PlotLumi/10TeV/ratios/WmL-WpL.pdf}
%\includegraphics[width=0.46\linewidth]{Notebooks/PlotLumi/10TeV/ratios/WmL-WpT.pdf}
%\includegraphics[width=0.46\linewidth]{Notebooks/PlotLumi/10TeV/ratios/WmT-WpL.pdf}
%\includegraphics[width=0.46\linewidth]{Notebooks/PlotLumi/10TeV/ratios/WmT-WpT.pdf}
%\includegraphics[width=0.46\linewidth]{Notebooks/PlotLumi/10TeV/ratios/Wmm-Wpm.pdf}
%\includegraphics[width=0.46\linewidth]{Notebooks/PlotLumi/10TeV/ratios/Wmm-Wpp.pdf}
%\end{figure}
%
%\begin{figure}[!t]
%\includegraphics[width=0.46\linewidth]{Notebooks/PlotLumi/10TeV/ratios/Wmp-Wpm.pdf}
%\includegraphics[width=0.46\linewidth]{Notebooks/PlotLumi/10TeV/ratios/Wmp-Wpp.pdf}

\begin{figure}[!t]
%\includegraphics[width=0.46\linewidth]{Notebooks/PlotLumi/10TeV/lumis/plotWWpolRandL.pdf}
%\includegraphics[width=0.46\linewidth]{Notebooks/PlotLumi/10TeV/lumis/plotWWpolTand0.pdf}
%\includegraphics[width=0.46\linewidth]{Notebooks/PlotLumi/10TeV/lumis/plotWmWpandWpWm.pdf}
\includegraphics[width=0.46\linewidth]{Notebooks/PlotLumi/10TeV/lumis/plotZZpolRandL.pdf}
\includegraphics[width=0.46\linewidth]{Notebooks/PlotLumi/10TeV/lumis/plotZZpolTand0.pdf}
\caption{Different luminosities depending on the polarization of Z \label{fig:lumiZZpol}}
\end{figure}





\begin{figure}[!t]
\includegraphics[width=0.46\linewidth]{Notebooks/PlotLumi/10TeV/ratios/ZL-ZL.pdf}
\includegraphics[width=0.46\linewidth]{Notebooks/PlotLumi/10TeV/ratios/ZL-ZT.pdf}
\includegraphics[width=0.46\linewidth]{Notebooks/PlotLumi/10TeV/ratios/ZT-ZL.pdf}
\includegraphics[width=0.46\linewidth]{Notebooks/PlotLumi/10TeV/ratios/ZT-ZT.pdf}
\caption{Different luminosities ratio over PDFs at central scale depending on the polarization of Z \label{fig:lumiratioZZpol}. In this sets of plot L=longitudinal, p=right, l=left.}
\end{figure}

\begin{figure}[!t]
\includegraphics[width=0.46\linewidth]{Notebooks/PlotLumi/10TeV/ratios/Zm-Zm.pdf}
\includegraphics[width=0.46\linewidth]{Notebooks/PlotLumi/10TeV/ratios/Zp-Zm.pdf}
\includegraphics[width=0.46\linewidth]{Notebooks/PlotLumi/10TeV/ratios/Zp-Zp.pdf}
\caption{Others like Fig.~\ref{fig:lumiratioZZpol}. \label{fig:lumiratioZZpol2}}
\end{figure}

\clearpage

\section{$WW$ initiated $t \bar t$ production}

We move now to the case of $WW$ initiated $t \bar t$ production. Here we repeat the exercise done in the previous section, but comparing ME given by $\mu^+e^-\to\bar \nu_\mu \nu_e t \bar t$ (setting the mass and the Yukawa of the electron to the muon one) and $WW\to t \bar t~$ with the EVA. Unlike the previous case, besides avoiding the contribution from $\mu^+\mu^-\to t \bar t Z (Z \nu_\mu \bar \nu_\mu )$ using the electron, we do not veto neither photons not other particles.

The equivalent plots of Figs.~\ref{fig:cutsZZtt} and \ref{fig:cutsZZtt2}, where the cuts are applied  are shown in Figs.~\ref{fig:cutsWWtt} and \ref{fig:cutsWWtt2}, while the  equivalent plots of Figs.~\ref{fig:nocutsZZtt} and \ref{fig:nocutsZZtt2}, cuts are NOT applied,  are shown in Figs.~\ref{fig:nocutsWWtt} and \ref{fig:nocutsWWtt2}. 


\begin{figure}[!t]
\includegraphics[width=0.46\linewidth]{Notebooks/PlotDistr/WW_tt/10TeVcuts/plotmtt.pdf}
\includegraphics[width=0.46\linewidth]{Notebooks/PlotDistr/WW_tt/10TeVcuts/plotmttratio1.pdf}
\includegraphics[width=0.46\linewidth]{Notebooks/PlotDistr/WW_tt/10TeVcuts/plotmttratio2.pdf}
\caption{With cuts, $m(t \bar t)$ distributions. \label{fig:cutsWWtt}}
\end{figure}


\begin{figure}[!t]
\includegraphics[width=0.46\linewidth]{Notebooks/PlotDistr/WW_tt/10TeVcuts/plotetat.pdf}
\includegraphics[width=0.46\linewidth]{Notebooks/PlotDistr/WW_tt/10TeVcuts/plotetatratio1.pdf}
\includegraphics[width=0.46\linewidth]{Notebooks/PlotDistr/WW_tt/10TeVcuts/plotetatratio2.pdf}
\caption{With cuts, $\eta(t)$ distributions. \label{fig:cutsWWtt2}}
\end{figure}
 
 
 \begin{figure}[!t]
\includegraphics[width=0.46\linewidth]{Notebooks/PlotDistr/WW_tt/10TeVnocuts/plotmtt.pdf}
\includegraphics[width=0.46\linewidth]{Notebooks/PlotDistr/WW_tt/10TeVnocuts/plotmttratio1.pdf}
\includegraphics[width=0.46\linewidth]{Notebooks/PlotDistr/WW_tt/10TeVnocuts/plotmttratio2.pdf}
\caption{No cuts, $m(t \bar t)$ distributions. \label{fig:nocutsWWtt}}
\end{figure}



\begin{figure}[!t]
\includegraphics[width=0.46\linewidth]{Notebooks/PlotDistr/WW_tt/10TeVnocuts/plotetat.pdf}
\includegraphics[width=0.46\linewidth]{Notebooks/PlotDistr/WW_tt/10TeVnocuts/plotetatratio1.pdf}
\includegraphics[width=0.46\linewidth]{Notebooks/PlotDistr/WW_tt/10TeVnocuts/plotetatratio2.pdf}
\caption{No cuts, $\eta(t)$ distributions. \label{fig:nocutsWWtt2}}
\end{figure}


Again we observe a good agreement for the invariant mass distribution when the cuts are applied and the scale is set in the EVA at $\mu=m(t \bar t)$, however now only for $m(t \bar t) \gtrsim 1 $ TeV. Also, in general the agreement is much worse than in the $ZZ$-initiated case and the contribution of the $WW$-initiated process is instead much larger. 

The equivalent of Fig.~\ref{fig:lumiZZpol} for the $WW$ initiated process is Fig.~\ref{fig:lumiWWpol} and the equivalent ratios as those in Figs.~\ref{fig:lumiratioZZpol} and \ref{fig:lumiratioZZpol2} are shown in Figs.~\ref{fig:lumiratioWWpol} and \ref{fig:lumiratioWWpol2}. In this case we can see how, for transverse polarisations, the difference between PDFs and EVA (resummation or higher-order effects) is larger than the PDFs scale uncertainty only for  $m(t \bar t) \gtrsim 5 $ TeV and it is anyway smaller than the difference between ME and EVA. The PDF vs EVA effects are larger for longitudinal $W$ bosons, but they are larger than the scale uncertainty only at  $m(t \bar t) \gtrsim 1 $ TeV, where such luminosity is already subdominant.  


\begin{figure}[!t]
\includegraphics[width=0.46\linewidth]{Notebooks/PlotLumi/10TeV/lumis/plotWWpolRandL.pdf}
\includegraphics[width=0.46\linewidth]{Notebooks/PlotLumi/10TeV/lumis/plotWWpolTand0.pdf}
%\includegraphics[width=0.46\linewidth]{Notebooks/PlotLumi/10TeV/lumis/plotWmWpandWpWm.pdf}
%\includegraphics[width=0.46\linewidth]{Notebooks/PlotLumi/10TeV/lumis/plotZZpolRandL.pdf}
%\includegraphics[width=0.46\linewidth]{Notebooks/PlotLumi/10TeV/lumis/plotZZpolTand0.pdf}
\caption{Different luminosities depending on the polarization of W \label{fig:lumiWWpol}}
\end{figure}
 
 
 \begin{figure}[!t]
\includegraphics[width=0.46\linewidth]{Notebooks/PlotLumi/10TeV/ratios/WmL-WpL.pdf}
\includegraphics[width=0.46\linewidth]{Notebooks/PlotLumi/10TeV/ratios/WmL-WpT.pdf}
\includegraphics[width=0.46\linewidth]{Notebooks/PlotLumi/10TeV/ratios/WmT-WpL.pdf}
\includegraphics[width=0.46\linewidth]{Notebooks/PlotLumi/10TeV/ratios/WmT-WpT.pdf}
\caption{Different luminosities ratio over PDFs at central scale depending on the polarization of Z \label{fig:lumiratioWWpol}. In this sets of plot L=longitudinal, p=right, l=left.}
\end{figure}

\begin{figure}[!t]
\includegraphics[width=0.46\linewidth]{Notebooks/PlotLumi/10TeV/ratios/Wmm-Wpm.pdf}
\includegraphics[width=0.46\linewidth]{Notebooks/PlotLumi/10TeV/ratios/Wmm-Wpp.pdf}
\includegraphics[width=0.46\linewidth]{Notebooks/PlotLumi/10TeV/ratios/Wmp-Wpm.pdf}
\caption{Others like Fig.~\ref{fig:lumiratioWWpol}. \label{fig:lumiratioWWpol2}}
\end{figure}


%
%\begin{figure}[!t]
%\includegraphics[width=0.46\linewidth]{Notebooks/PlotLumi/10TeV/ratios/ZL-ZL.pdf}
%\includegraphics[width=0.46\linewidth]{Notebooks/PlotLumi/10TeV/ratios/ZL-ZT.pdf}
%\includegraphics[width=0.46\linewidth]{Notebooks/PlotLumi/10TeV/ratios/ZT-ZL.pdf}
%\includegraphics[width=0.46\linewidth]{Notebooks/PlotLumi/10TeV/ratios/ZT-ZT.pdf}
%\caption{Different luminosities ratio over PDFs at central scale depending on the polarization of Z \label{fig:lumiratioZZpol}. In this sets of plot L=longitudinal, p=right, l=left.}
%\end{figure}
%
%\begin{figure}[!t]
%\includegraphics[width=0.46\linewidth]{Notebooks/PlotLumi/10TeV/ratios/Zm-Zm.pdf}
%\includegraphics[width=0.46\linewidth]{Notebooks/PlotLumi/10TeV/ratios/Zp-Zm.pdf}
%\includegraphics[width=0.46\linewidth]{Notebooks/PlotLumi/10TeV/ratios/Zp-Zp.pdf}
%\caption{Others like Fig.~\ref{fig:lumiratioZZpol}. \label{fig:lumiratioZZpol2}}
%\end{figure}


\clearpage

\subsection{$WW$ into $WW$}

We move now to $WW$ production ($W^+W^-$), which as we will show is a very different case w.r.t.~the $t \bar t~$ production.
First of all we notice that we cannot easily isolate the $ZZ\to WW$ component in the ME via vetoing photons in $\mu^+e^-\to\mu^+e^- WW$ production, such operation violate gauge invariance and leads in numerical simulation to bogus numbers. We therefore perform this test only for the charged channel, $WW\to WW$ and its ME counterpart following the same logic of the case of $t \bar t$ production: selecting $\mu^+e^-\to\bar \nu_\mu \nu_e WW$.


\begin{figure}[!t]
\includegraphics[width=0.46\linewidth]{Notebooks/PlotDistr/WW_WW/10TeVolnlyptcut/plotmWW.pdf}
\includegraphics[width=0.46\linewidth]{Notebooks/PlotDistr/WW_WW/10TeVolnlyptcut/plotmWWratio1.pdf}
\includegraphics[width=0.46\linewidth]{Notebooks/PlotDistr/WW_WW/10TeVolnlyptcut/plotmWWratio2.pdf}
\caption{$p_T(W)$ cut only, $m(WW)$ distributions. \label{fig:ptcutWWWW}}

\end{figure}

\begin{figure}[!t]
\includegraphics[width=0.46\linewidth]{Notebooks/PlotDistr/WW_WW/10TeVolnlyptcut/plotetaW.pdf}
\includegraphics[width=0.46\linewidth]{Notebooks/PlotDistr/WW_WW/10TeVolnlyptcut/plotetaWratio1.pdf}
\includegraphics[width=0.46\linewidth]{Notebooks/PlotDistr/WW_WW/10TeVolnlyptcut/plotetaWratio2.pdf}
\caption{$p_T(W)$ cut only, $\eta(W)$ distributions. \label{fig:ptcutWWWW2}}
\end{figure}


In the case of $WW \to WW$ (EVA) vs $\mu^+e^-\to\bar \nu_\mu \nu_e WW$ (ME) there is an important point to note: with no cuts on the $W$ bosons the former diverges and the latter does not. The problem in $WW\to WW$ is not due to the EVA but to the partonic cross section, which is divergent! This is the same effect observed in Bhabha scattering with not restriction on the angle of the final state particles w.r.t.~the beam. The same process within the $\mu^+e^-\to\bar \nu_\mu \nu_e WW$ ME has an important difference: the $W$ bosons in the initial state are space-like and this condition automatically screens the appearance of the divergence. All in all, this tell us that we have already identified a situation, which is not very exotic being the inclusive cross section, where the EVA is badly failing. Nevertheless, we continue our analysis by requiring a minimum $p_T$ for the W bosons, $p_T(W)>150$ GeV, so that the EVA is finite, and we show the relevant plots as done in the previous sections in Figs.~\ref{fig:ptcutWWWW} and \ref{fig:ptcutWWWW2}. 

The prediction of the EVA are completely off from the ME, overshooting the result. This is not surprising since with a cut  $p_T(W)>p_{T,cut}$ in the limit $p_{T,cut}\to 0$ the EVA diverges while the ME prediction remain finite. Moreover, for a given $p_{T,cut}$ the corresponding angular cut on the $\theta$ of the outgoing $W$ bosons decreases for larger $m(WW)$, leading to a larger discrepancy between ME and EVA. \pnote{statements to check numerically by varying the cut}

A way to avoid small $\theta$ angles is applying a cut on $\eta(W)$. We therefore apply similar cuts to those of the $t \bar t$ case 
%
\begin{equation}
m(WW) > 500~\gev, ~~p_T(W) > 150~\gev,~~ |\eta(W)|<2.5 
\end{equation}
 %
 and we show the corresponding results in Figs.~\ref{fig:cutsWWWW} and \ref{fig:cutsWWWW2}.




\begin{figure}[!t]
\includegraphics[width=0.46\linewidth]{Notebooks/PlotDistr/WW_WW/10TeVcuts/plotmWW.pdf}
\includegraphics[width=0.46\linewidth]{Notebooks/PlotDistr/WW_WW/10TeVcuts/plotmWWratio1.pdf}
\includegraphics[width=0.46\linewidth]{Notebooks/PlotDistr/WW_WW/10TeVcuts/plotmWWratio2.pdf}
\caption{With cuts, $m(WW)$ distributions. \label{fig:cutsWWWW}}
\end{figure}



\begin{figure}[!t]
\includegraphics[width=0.46\linewidth]{Notebooks/PlotDistr/WW_WW/10TeVcuts/plotetaW.pdf}
\includegraphics[width=0.46\linewidth]{Notebooks/PlotDistr/WW_WW/10TeVcuts/plotetaWratio1.pdf}
\includegraphics[width=0.46\linewidth]{Notebooks/PlotDistr/WW_WW/10TeVcuts/plotetaWratio2.pdf}
\caption{With cuts, $\eta(W)$ distributions. \label{fig:cutsWWWW2}}
\end{figure}


Now the ME are in agreement with the EVA uncertainty band, not in the case ``EVA only Log[mu/MV]'', but this band involves a variation of order $+80\%$ and $-60\%$, which is much larger than the differences between the EVA and the PDF luminosities, which have already been shown in Figs.~\ref{fig:lumiWWpol}--\ref{fig:lumiratioWWpol2}.

We notice also that within the $\mu^+e^-\to\bar \nu_\mu \nu_e WW$ ME, there are other contributions that are logarithmically enhanced, for instance,
\begin{itemize}
\item via the radiation of $W$ bosons in the initial state $\bar \nu_\mu \nu_e \to \bar \nu_\mu \nu_e $.
\item via the radiation of one $W$  and one $\nu$ in the initial state $\nu W \to \nu W  $.
\item radiation of $W$ bosons in the final state from the process  $\mu^+e^-\to \mu^+e^-$.
\end{itemize}

The cuts applied should suppress such contributions, but a component should still survive, especially if only the $p_T(W)$ cut or no cuts at all are applied, which is a physically well defined configuration that as we have already said can only be performed via ME.

In addition we remind the reader that we are simulating ME for  $\mu^+e^-\to\bar \nu_\mu \nu_e WW$, if we had chosen $\mu^+\mu^-\to\bar \nu_\mu \nu_\mu WW$ additional new topologies would appear, such as

\begin{itemize}
\item $\mu^+\mu^-\to ZH (Z\to \bar \nu_\mu \nu_\mu ) (H \to WW )$
\item $\mu^+\mu^-\to ZWW (Z\to \bar \nu_\mu \nu_\mu ) $
\item $\mu^+\mu^-\to Z_1Z_2 (Z\to \bar \nu_\mu \nu_\mu ) (Z_2 \to WW) $
\item $\mu^+\mu^-\to Z \gamma (Z\to \bar \nu_\mu \nu_\mu ) (\gamma \to WW) $
\end{itemize}

None of them is VBF-like and therefore the agreement with the EVA could get only worse.

As last study we try to find a situation where the EVA could work. We take a $(\rm hypothetical)^2$ collider: hypothetical being a muon collider and hypothetical because is at 100 TeV, something that could be built in the next century. We also increase by a factor of 10 the $m(WW)$ and $p_T(W)$ cuts 
\begin{equation}
m(WW) > 5~\tev, ~~p_T(W) > 1.5~\tev,~~ |\eta(W)|<2.5 \,.
\end{equation}

Results are shown in Fig.~



\begin{figure}[!t]
\includegraphics[width=0.46\linewidth]{Notebooks/PlotDistr/WW_WW/100TeVcuts/plotmWW.pdf}
\includegraphics[width=0.46\linewidth]{Notebooks/PlotDistr/WW_WW/100TeVcuts/plotmWWratio1.pdf}
\includegraphics[width=0.46\linewidth]{Notebooks/PlotDistr/WW_WW/100TeVcuts/plotmWWratio2.pdf}
\caption{100 TeV with cuts, $m(WW)$ distributions. \label{fig:100cutsWWWW}}
\end{figure}



\begin{figure}[!t]
\includegraphics[width=0.46\linewidth]{Notebooks/PlotDistr/WW_WW/100TeVcuts/plotetaW.pdf}
\includegraphics[width=0.46\linewidth]{Notebooks/PlotDistr/WW_WW/100TeVcuts/plotetaWratio1.pdf}
\includegraphics[width=0.46\linewidth]{Notebooks/PlotDistr/WW_WW/100TeVcuts/plotetaWratio2.pdf}
\caption{100 TeV with cuts, $\eta(W)$ distributions. \label{fig:100cutsWWWW2}}
\end{figure}






\clearpage
\pagebreak

%%%%%%%%%%%%%%%%%%%%%%%%%%%%%%%%%%%%%%
%%%%%%%%%%%%%%%%%%%%%%%%%%%%%%%%%%%%%%
%%%%%%%%%%%%%%%%%%%%%%%%%%%%%%%%%%%%%%
%%%%%%%%%%%%%%%%%%%%%%%%%%%%%%%%%%%%%%
%%%%%%%%%%%%%%%%%%%%%%%%%%%%%%%%%%%%%%
%%%%%%%%%%%%%%%%%%%%%%%%%%%%%%%%%%%%%%
%%%%%%%%%%%%%%%%%%%%%%%%%%%%%%%%%%%%%%
%%%%%%%%%%%%%%%%%%%%%%%%%%%%%%%%%%%%%%
%%%%%%%%%%%%%%%%%%%%%%%%%%%%%%%%%%%%%%
%%%%%%%%%%%%%%%%%%%%%%%%%%%%%%%%%%%%%%
%%%%%%%%%%%%%%%%%%%%%%%%%%%%%%%%%%%%%%
%%%%%%%%%%%%%%%%%%%%%%%%%%%%%%%%%%%%%%
%%%%%%%%%%%%%%%%%%%%%%%%%%%%%%%%%%%%%%
%%%%%%%%%%%%%%%%%%%%%%%%%%%%%%%%%%%%%%
%%%%%%%%%%%%%%%%%%%%%%%%%%%%%%%%%%%%%%
%%%%%%%%%%%%%%%%%%%%%%%%%%%%%%%%%%%%%%
%%%%%%%%%%%%%%%%%%%%%%%%%%%%%%%%%%%%%%
%%%%%%%%%%%%%%%%%%%%%%%%%%%%%%%%%%%%%%
%%%%%%%%%%%%%%%%%%%%%%%%%%%%%%%%%%%%%%
%%%%%%%%%%%%%%%%%%%%%%%%%%%%%%%%%%%%%%
%%%%%%%%%%%%%%%%%%%%%%%%%%%%%%%%%%%%%%
%%%%%%%%%%%%%%%%%%%%%%%%%%%%%%%%%%%%%%
%%%%%%%%%%%%%%%%%%%%%%%%%%%%%%%%%%%%%%





\appendix

\clearpage

\section{PDF vs EVA}
\label{app:PDFsEVA}

\subsection{$Z$ PDF}
\label{app:PDFsEVA_Z}


Here we show for the individual PDFs, with different polarisation the differences with EVA. In each Figure we show for three different scale choices ($Q=\sqrt{s}, \sqrt{s} \sqrt{x}$ and $\sqrt{s} x$) the ratio with the PDF at the reference scale, minus 1, for the EVA (orange) and the ``EVA only Log[mu/MV]'' (blue). We also show in black the ratio, minus 1, for the PDFs itself varying only in the numerator by a factor of two the scale



\begin{figure}[!b]
\includegraphics[width=0.46\linewidth]{Notebooks/PlotPDFs/ratios/10TeV/Z_0_Q.pdf}
\includegraphics[width=0.46\linewidth]{Notebooks/PlotPDFs/ratios/10TeV/Z_0_Qsqrtx.pdf}
\includegraphics[width=0.46\linewidth]{Notebooks/PlotPDFs/ratios/10TeV/Z_0_Qx.pdf}
\caption{$Z$ longitudinal}
\end{figure}

\begin{figure}[!b]
\includegraphics[width=0.46\linewidth]{Notebooks/PlotPDFs/ratios/10TeV/Z_L-_Q.pdf}
\includegraphics[width=0.46\linewidth]{Notebooks/PlotPDFs/ratios/10TeV/Z_L-_Qsqrtx.pdf}
\includegraphics[width=0.46\linewidth]{Notebooks/PlotPDFs/ratios/10TeV/Z_L-_Qx.pdf}
\caption{$Z$ left}
\end{figure}

\begin{figure}[!b]
\includegraphics[width=0.46\linewidth]{Notebooks/PlotPDFs/ratios/10TeV/Z_R_Q.pdf}
\includegraphics[width=0.46\linewidth]{Notebooks/PlotPDFs/ratios/10TeV/Z_R_Qsqrtx.pdf}
\includegraphics[width=0.46\linewidth]{Notebooks/PlotPDFs/ratios/10TeV/Z_R_Qx.pdf}
\caption{$Z$ right}
\end{figure}

\begin{figure}[!b]
\includegraphics[width=0.46\linewidth]{Notebooks/PlotPDFs/ratios/10TeV/Z_T_Q.pdf}
\includegraphics[width=0.46\linewidth]{Notebooks/PlotPDFs/ratios/10TeV/Z_T_Qsqrtx.pdf}
\includegraphics[width=0.46\linewidth]{Notebooks/PlotPDFs/ratios/10TeV/Z_T_Qx.pdf}
\caption{$Z$ transverse}
\end{figure}


\clearpage



\subsection{W PDF}
\label{app:PDFsEVA_W}

Same as the $Z$, but for the $W$.


\begin{figure}[!b]
\includegraphics[width=0.46\linewidth]{Notebooks/PlotPDFs/ratios/10TeV/W-_0_Q.pdf}
\includegraphics[width=0.46\linewidth]{Notebooks/PlotPDFs/ratios/10TeV/W-_0_Qsqrtx.pdf}
\includegraphics[width=0.46\linewidth]{Notebooks/PlotPDFs/ratios/10TeV/W-_0_Qx.pdf}
\caption{$W^-$ longitudinal}
\end{figure}

\begin{figure}[!b]
\includegraphics[width=0.46\linewidth]{Notebooks/PlotPDFs/ratios/10TeV/W-_L_Q.pdf}
\includegraphics[width=0.46\linewidth]{Notebooks/PlotPDFs/ratios/10TeV/W-_L_Qsqrtx.pdf}
\includegraphics[width=0.46\linewidth]{Notebooks/PlotPDFs/ratios/10TeV/W-_L_Qx.pdf}
\caption{$W^-$ left}
\end{figure}

\begin{figure}[!b]
\includegraphics[width=0.46\linewidth]{Notebooks/PlotPDFs/ratios/10TeV/W-_R_Q.pdf}
\includegraphics[width=0.46\linewidth]{Notebooks/PlotPDFs/ratios/10TeV/W-_R_Qsqrtx.pdf}
\includegraphics[width=0.46\linewidth]{Notebooks/PlotPDFs/ratios/10TeV/W-_R_Qx.pdf}
\caption{$W^-$ right}
\end{figure}

\begin{figure}[!b]
\includegraphics[width=0.46\linewidth]{Notebooks/PlotPDFs/ratios/10TeV/W-_T_Q.pdf}
\includegraphics[width=0.46\linewidth]{Notebooks/PlotPDFs/ratios/10TeV/W-_T_Qsqrtx.pdf}
\includegraphics[width=0.46\linewidth]{Notebooks/PlotPDFs/ratios/10TeV/W-_T_Qx.pdf}
\caption{$W^-$ transverse}
\end{figure}




%%%%%%%%%%%%%%%%%%%%%%%%%%%%%%%%%%%%%%
%%%%%%%%%%%%%%%%%%%%%%%%%%%%%%%%%%%%%%
%%%%%%%%%%%%%%%%%%%%%%%%%%%%%%%%%%%%%%
%%%%%%%%%%%%%%%%%%%%%%%%%%%%%%%%%%%%%%
%%%%%%%%%%%%%%%%%%%%%%%%%%%%%%%%%%%%%%
%%%%%%%%%%%%%%%%%%%%%%%%%%%%%%%%%%%%%%
%%%%%%%%%%%%%%%%%%%%%%%%%%%%%%%%%%%%%%
%%%%%%%%%%%%%%%%%%%%%%%%%%%%%%%%%%%%%%
%%%%%%%%%%%%%%%%%%%%%%%%%%%%%%%%%%%%%%
%%%%%%%%%%%%%%%%%%%%%%%%%%%%%%%%%%%%%%
%%%%%%%%%%%%%%%%%%%%%%%%%%%%%%%%%%%%%%
%%%%%%%%%%%%%%%%%%%%%%%%%%%%%%%%%%%%%%
%%%%%%%%%%%%%%%%%%%%%%%%%%%%%%%%%%%%%%
%%%%%%%%%%%%%%%%%%%%%%%%%%%%%%%%%%%%%%
%%%%%%%%%%%%%%%%%%%%%%%%%%%%%%%%%%%%%%
%%%%%%%%%%%%%%%%%%%%%%%%%%%%%%%%%%%%%%
%%%%%%%%%%%%%%%%%%%%%%%%%%%%%%%%%%%%%%
%%%%%%%%%%%%%%%%%%%%%%%%%%%%%%%%%%%%%%
%%%%%%%%%%%%%%%%%%%%%%%%%%%%%%%%%%%%%%
%%%%%%%%%%%%%%%%%%%%%%%%%%%%%%%%%%%%%%
%%%%%%%%%%%%%%%%%%%%%%%%%%%%%%%%%%%%%%
%%%%%%%%%%%%%%%%%%%%%%%%%%%%%%%%%%%%%%
%%%%%%%%%%%%%%%%%%%%%%%%%%%%%%%%%%%%%%


\section{ME vs EVA}

\subsection{$WW$ into $WW$}


%\begin{figure}[!t]
%\includegraphics[width=0.46\linewidth]{Notebooks/PlotDistr/WW_WW/100TeVcuts/plotetaW.pdf}
%\includegraphics[width=0.46\linewidth]{Notebooks/PlotDistr/WW_WW/100TeVcuts/plotetaWratio1.pdf}
%\includegraphics[width=0.46\linewidth]{Notebooks/PlotDistr/WW_WW/100TeVcuts/plotetaWratio2.pdf}
%\end{figure}

%\begin{figure}[!t]
%\includegraphics[width=0.46\linewidth]{Notebooks/PlotDistr/WW_WW/100TeVcuts/plotmWW.pdf}
%\includegraphics[width=0.46\linewidth]{Notebooks/PlotDistr/WW_WW/100TeVcuts/plotmWWratio1.pdf}
%\includegraphics[width=0.46\linewidth]{Notebooks/PlotDistr/WW_WW/100TeVcuts/plotmWWratio2.pdf}
%\end{figure}

%\begin{figure}[!t]
%\includegraphics[width=0.46\linewidth]{Notebooks/PlotDistr/WW_WW/10TeVcuts/plotetaW.pdf}
%\includegraphics[width=0.46\linewidth]{Notebooks/PlotDistr/WW_WW/10TeVcuts/plotetaWratio1.pdf}
%\includegraphics[width=0.46\linewidth]{Notebooks/PlotDistr/WW_WW/10TeVcuts/plotetaWratio2.pdf}
%\end{figure}

%\begin{figure}[!t]
%\includegraphics[width=0.46\linewidth]{Notebooks/PlotDistr/WW_WW/10TeVcuts/plotmWW.pdf}
%\includegraphics[width=0.46\linewidth]{Notebooks/PlotDistr/WW_WW/10TeVcuts/plotmWWratio1.pdf}
%\includegraphics[width=0.46\linewidth]{Notebooks/PlotDistr/WW_WW/10TeVcuts/plotmWWratio2.pdf}
%\end{figure}

%\begin{figure}[!t]
%\includegraphics[width=0.46\linewidth]{Notebooks/PlotDistr/WW_WW/10TeVolnlyptcut/plotetaW.pdf}
%\includegraphics[width=0.46\linewidth]{Notebooks/PlotDistr/WW_WW/10TeVolnlyptcut/plotetaWratio1.pdf}
%\includegraphics[width=0.46\linewidth]{Notebooks/PlotDistr/WW_WW/10TeVolnlyptcut/plotetaWratio2.pdf}
%\end{figure}

%\begin{figure}[!t]
%\includegraphics[width=0.46\linewidth]{Notebooks/PlotDistr/WW_WW/10TeVolnlyptcut/plotmWW.pdf}
%\includegraphics[width=0.46\linewidth]{Notebooks/PlotDistr/WW_WW/10TeVolnlyptcut/plotmWWratio1.pdf}
%\includegraphics[width=0.46\linewidth]{Notebooks/PlotDistr/WW_WW/10TeVolnlyptcut/plotmWWratio2.pdf}
%\end{figure}

\clearpage
\subsection{$WW$ into $t \bar t$}

%\begin{figure}[!t]
%\includegraphics[width=0.46\linewidth]{Notebooks/PlotDistr/WW_tt/10TeVcuts/plotetat.pdf}
%\includegraphics[width=0.46\linewidth]{Notebooks/PlotDistr/WW_tt/10TeVcuts/plotetatratio1.pdf}
%\includegraphics[width=0.46\linewidth]{Notebooks/PlotDistr/WW_tt/10TeVcuts/plotetatratio2.pdf}
%\end{figure}

%\begin{figure}[!t]
%\includegraphics[width=0.46\linewidth]{Notebooks/PlotDistr/WW_tt/10TeVcuts/plotmtt.pdf}
%\includegraphics[width=0.46\linewidth]{Notebooks/PlotDistr/WW_tt/10TeVcuts/plotmttratio1.pdf}
%\includegraphics[width=0.46\linewidth]{Notebooks/PlotDistr/WW_tt/10TeVcuts/plotmttratio2.pdf}
%\end{figure}

%\begin{figure}[!t]
%\includegraphics[width=0.46\linewidth]{Notebooks/PlotDistr/WW_tt/10TeVnocuts/plotetat.pdf}
%\includegraphics[width=0.46\linewidth]{Notebooks/PlotDistr/WW_tt/10TeVnocuts/plotetatratio1.pdf}
%\includegraphics[width=0.46\linewidth]{Notebooks/PlotDistr/WW_tt/10TeVnocuts/plotetatratio2.pdf}
%\end{figure}

%\begin{figure}[!t]
%\includegraphics[width=0.46\linewidth]{Notebooks/PlotDistr/WW_tt/10TeVnocuts/plotmtt.pdf}
%\includegraphics[width=0.46\linewidth]{Notebooks/PlotDistr/WW_tt/10TeVnocuts/plotmttratio1.pdf}
%\includegraphics[width=0.46\linewidth]{Notebooks/PlotDistr/WW_tt/10TeVnocuts/plotmttratio2.pdf}
%\end{figure}

\clearpage
\subsection{$ZZ$ into $t \bar t$}

%\begin{figure}[!t]
%\includegraphics[width=0.46\linewidth]{Notebooks/PlotDistr/ZZ_tt/10TeVcuts/plotetat.pdf}
%\includegraphics[width=0.46\linewidth]{Notebooks/PlotDistr/ZZ_tt/10TeVcuts/plotetatratio1.pdf}
%\includegraphics[width=0.46\linewidth]{Notebooks/PlotDistr/ZZ_tt/10TeVcuts/plotetatratio2.pdf}
%\end{figure}

%\begin{figure}[!t]
%\includegraphics[width=0.46\linewidth]{Notebooks/PlotDistr/ZZ_tt/10TeVcuts/plotmtt.pdf}
%\includegraphics[width=0.46\linewidth]{Notebooks/PlotDistr/ZZ_tt/10TeVcuts/plotmttratio1.pdf}
%\includegraphics[width=0.46\linewidth]{Notebooks/PlotDistr/ZZ_tt/10TeVcuts/plotmttratio2.pdf}
%\end{figure}

%\begin{figure}[!t]
%\includegraphics[width=0.46\linewidth]{Notebooks/PlotDistr/ZZ_tt/10TeVnocuts/plotetat.pdf}
%\includegraphics[width=0.46\linewidth]{Notebooks/PlotDistr/ZZ_tt/10TeVnocuts/plotetatratio1.pdf}
%\includegraphics[width=0.46\linewidth]{Notebooks/PlotDistr/ZZ_tt/10TeVnocuts/plotetatratio2.pdf}
%\end{figure}

%\begin{figure}[!t]
%\includegraphics[width=0.46\linewidth]{Notebooks/PlotDistr/ZZ_tt/10TeVnocuts/plotmtt.pdf}
%\includegraphics[width=0.46\linewidth]{Notebooks/PlotDistr/ZZ_tt/10TeVnocuts/plotmttratio1.pdf}
%\includegraphics[width=0.46\linewidth]{Notebooks/PlotDistr/ZZ_tt/10TeVnocuts/plotmttratio2.pdf}
%\end{figure}

%%\begin{figure}[!t]
%%\includegraphics[width=0.46\linewidth]{Notebooks/PlotLumi/10TeV/lumis/WmL-WpL.pdf}
%%\includegraphics[width=0.46\linewidth]{Notebooks/PlotLumi/10TeV/lumis/WmL-WpT.pdf}
%%\includegraphics[width=0.46\linewidth]{Notebooks/PlotLumi/10TeV/lumis/WmT-WpL.pdf}
%%\includegraphics[width=0.46\linewidth]{Notebooks/PlotLumi/10TeV/lumis/WmT-WpT.pdf}
%%\includegraphics[width=0.46\linewidth]{Notebooks/PlotLumi/10TeV/lumis/Wmm-Wpm.pdf}
%%\includegraphics[width=0.46\linewidth]{Notebooks/PlotLumi/10TeV/lumis/Wmm-Wpp.pdf}
%%\end{figure}
%
%%\begin{figure}[!t]
%%\includegraphics[width=0.46\linewidth]{Notebooks/PlotLumi/10TeV/lumis/Wmp-Wpm.pdf}
%%\includegraphics[width=0.46\linewidth]{Notebooks/PlotLumi/10TeV/lumis/Wmp-Wpp.pdf}
%%\includegraphics[width=0.46\linewidth]{Notebooks/PlotLumi/10TeV/lumis/ZL-ZL.pdf}
%%\includegraphics[width=0.46\linewidth]{Notebooks/PlotLumi/10TeV/lumis/ZL-ZT.pdf}
%%\includegraphics[width=0.46\linewidth]{Notebooks/PlotLumi/10TeV/lumis/ZT-ZL.pdf}
%%\includegraphics[width=0.46\linewidth]{Notebooks/PlotLumi/10TeV/lumis/ZT-ZT.pdf}
%%\end{figure}
%
%%\begin{figure}[!t]
%%\includegraphics[width=0.46\linewidth]{Notebooks/PlotLumi/10TeV/lumis/Zm-Zm.pdf}
%%\includegraphics[width=0.46\linewidth]{Notebooks/PlotLumi/10TeV/lumis/Zp-Zm.pdf}
%%\includegraphics[width=0.46\linewidth]{Notebooks/PlotLumi/10TeV/lumis/Zp-Zp.pdf}
%%\end{figure}
%

%%\clearpage
%%\section{Luminosities}
%%
%%\subsection{10 TeV}
%%
%%%\begin{figure}[!t]
%%%\includegraphics[width=0.46\linewidth]{Notebooks/PlotLumi/10TeV/lumis/plotWWpolRandL.pdf}
%%%\includegraphics[width=0.46\linewidth]{Notebooks/PlotLumi/10TeV/lumis/plotWWpolTand0.pdf}
%%%\includegraphics[width=0.46\linewidth]{Notebooks/PlotLumi/10TeV/lumis/plotWmWpandWpWm.pdf}
%%%\includegraphics[width=0.46\linewidth]{Notebooks/PlotLumi/10TeV/lumis/plotZZpolRandL.pdf}
%%%\includegraphics[width=0.46\linewidth]{Notebooks/PlotLumi/10TeV/lumis/plotZZpolTand0.pdf}
%%%\end{figure}
%%
%%%\begin{figure}
%%%\includegraphics[width=0.46\linewidth]{Notebooks/PlotLumi/10TeV/lumis/plotgammaWZ.pdf}
%%%\end{figure}
%%
%%\clearpage
%%\subsubsection{Comparisons with EVAs}
%%
%%%\begin{figure}[!t]
%%%\includegraphics[width=0.46\linewidth]{Notebooks/PlotLumi/10TeV/ratios/WmL-WpL.pdf}
%%%\includegraphics[width=0.46\linewidth]{Notebooks/PlotLumi/10TeV/ratios/WmL-WpT.pdf}
%%%\includegraphics[width=0.46\linewidth]{Notebooks/PlotLumi/10TeV/ratios/WmT-WpL.pdf}
%%%\includegraphics[width=0.46\linewidth]{Notebooks/PlotLumi/10TeV/ratios/WmT-WpT.pdf}
%%%\includegraphics[width=0.46\linewidth]{Notebooks/PlotLumi/10TeV/ratios/Wmm-Wpm.pdf}
%%%\includegraphics[width=0.46\linewidth]{Notebooks/PlotLumi/10TeV/ratios/Wmm-Wpp.pdf}
%%%\end{figure}
%%
%%%\begin{figure}[!t]
%%%\includegraphics[width=0.46\linewidth]{Notebooks/PlotLumi/10TeV/ratios/Wmp-Wpm.pdf}
%%%\includegraphics[width=0.46\linewidth]{Notebooks/PlotLumi/10TeV/ratios/Wmp-Wpp.pdf}
%%%\includegraphics[width=0.46\linewidth]{Notebooks/PlotLumi/10TeV/ratios/ZL-ZL.pdf}
%%%\includegraphics[width=0.46\linewidth]{Notebooks/PlotLumi/10TeV/ratios/ZL-ZT.pdf}
%%%\includegraphics[width=0.46\linewidth]{Notebooks/PlotLumi/10TeV/ratios/ZT-ZL.pdf}
%%%\includegraphics[width=0.46\linewidth]{Notebooks/PlotLumi/10TeV/ratios/ZT-ZT.pdf}
%%%\end{figure}
%%
%%%\begin{figure}[!t]
%%%\includegraphics[width=0.46\linewidth]{Notebooks/PlotLumi/10TeV/ratios/Zm-Zm.pdf}
%%%\includegraphics[width=0.46\linewidth]{Notebooks/PlotLumi/10TeV/ratios/Zp-Zm.pdf}
%%%\includegraphics[width=0.46\linewidth]{Notebooks/PlotLumi/10TeV/ratios/Zp-Zp.pdf}
%%%\end{figure}
%%
%%%%\begin{figure}[!t]
%%%%\includegraphics[width=0.46\linewidth]{Notebooks/PlotLumi/1TeV/lumis/WmL-WpL.pdf}
%%%%\includegraphics[width=0.46\linewidth]{Notebooks/PlotLumi/1TeV/lumis/WmL-WpT.pdf}
%%%%\includegraphics[width=0.46\linewidth]{Notebooks/PlotLumi/1TeV/lumis/WmT-WpL.pdf}
%%%%\includegraphics[width=0.46\linewidth]{Notebooks/PlotLumi/1TeV/lumis/WmT-WpT.pdf}
%%%%\includegraphics[width=0.46\linewidth]{Notebooks/PlotLumi/1TeV/lumis/Wmm-Wpm.pdf}
%%%%\includegraphics[width=0.46\linewidth]{Notebooks/PlotLumi/1TeV/lumis/Wmm-Wpp.pdf}
%%%%\end{figure}
%%%
%%%%\begin{figure}[!t]
%%%%\includegraphics[width=0.46\linewidth]{Notebooks/PlotLumi/1TeV/lumis/Wmp-Wpm.pdf}
%%%%\includegraphics[width=0.46\linewidth]{Notebooks/PlotLumi/1TeV/lumis/Wmp-Wpp.pdf}
%%%%\includegraphics[width=0.46\linewidth]{Notebooks/PlotLumi/1TeV/lumis/ZL-ZL.pdf}
%%%%\includegraphics[width=0.46\linewidth]{Notebooks/PlotLumi/1TeV/lumis/ZL-ZT.pdf}
%%%%\includegraphics[width=0.46\linewidth]{Notebooks/PlotLumi/1TeV/lumis/ZT-ZL.pdf}
%%%%\includegraphics[width=0.46\linewidth]{Notebooks/PlotLumi/1TeV/lumis/ZT-ZT.pdf}
%%%%\end{figure}
%%%
%%%%\begin{figure}[!t]
%%%%\includegraphics[width=0.46\linewidth]{Notebooks/PlotLumi/1TeV/lumis/Zm-Zm.pdf}
%%%%\includegraphics[width=0.46\linewidth]{Notebooks/PlotLumi/1TeV/lumis/Zp-Zm.pdf}
%%%%\includegraphics[width=0.46\linewidth]{Notebooks/PlotLumi/1TeV/lumis/Zp-Zp.pdf}
%%%%\end{figure}
%%%
%%\clearpage
%%\subsection{1 TeV}
%%
%%%\begin{figure}[!t]
%%%\includegraphics[width=0.46\linewidth]{Notebooks/PlotLumi/1TeV/lumis/plotWWpolRandL.pdf}
%%%\includegraphics[width=0.46\linewidth]{Notebooks/PlotLumi/1TeV/lumis/plotWWpolTand0.pdf}
%%%\includegraphics[width=0.46\linewidth]{Notebooks/PlotLumi/1TeV/lumis/plotWmWpandWpWm.pdf}
%%%\includegraphics[width=0.46\linewidth]{Notebooks/PlotLumi/1TeV/lumis/plotZZpolRandL.pdf}
%%%\includegraphics[width=0.46\linewidth]{Notebooks/PlotLumi/1TeV/lumis/plotZZpolTand0.pdf}
%%%\end{figure}
%%
%%%\begin{figure}
%%%\includegraphics[width=0.46\linewidth]{Notebooks/PlotLumi/1TeV/lumis/plotgammaWZ.pdf}
%%%\end{figure}
%%
%%\clearpage
%%\subsubsection{Comparisons with EVAs}
%%
%%
%%%\begin{figure}[!t]
%%%\includegraphics[width=0.46\linewidth]{Notebooks/PlotLumi/1TeV/ratios/WmL-WpL.pdf}
%%%\includegraphics[width=0.46\linewidth]{Notebooks/PlotLumi/1TeV/ratios/WmL-WpT.pdf}
%%%\includegraphics[width=0.46\linewidth]{Notebooks/PlotLumi/1TeV/ratios/WmT-WpL.pdf}
%%%\includegraphics[width=0.46\linewidth]{Notebooks/PlotLumi/1TeV/ratios/WmT-WpT.pdf}
%%%\includegraphics[width=0.46\linewidth]{Notebooks/PlotLumi/1TeV/ratios/Wmm-Wpm.pdf}
%%%\includegraphics[width=0.46\linewidth]{Notebooks/PlotLumi/1TeV/ratios/Wmm-Wpp.pdf}
%%%\end{figure}
%%
%%%\begin{figure}[!t]
%%%\includegraphics[width=0.46\linewidth]{Notebooks/PlotLumi/1TeV/ratios/Wmp-Wpm.pdf}
%%%\includegraphics[width=0.46\linewidth]{Notebooks/PlotLumi/1TeV/ratios/Wmp-Wpp.pdf}
%%%\includegraphics[width=0.46\linewidth]{Notebooks/PlotLumi/1TeV/ratios/ZL-ZL.pdf}
%%%\includegraphics[width=0.46\linewidth]{Notebooks/PlotLumi/1TeV/ratios/ZL-ZT.pdf}
%%%\includegraphics[width=0.46\linewidth]{Notebooks/PlotLumi/1TeV/ratios/ZT-ZL.pdf}
%%%\includegraphics[width=0.46\linewidth]{Notebooks/PlotLumi/1TeV/ratios/ZT-ZT.pdf}
%%%\end{figure}
%%
%%%\begin{figure}[!t]
%%%\includegraphics[width=0.46\linewidth]{Notebooks/PlotLumi/1TeV/ratios/Zm-Zm.pdf}
%%%\includegraphics[width=0.46\linewidth]{Notebooks/PlotLumi/1TeV/ratios/Zp-Zm.pdf}
%%%\includegraphics[width=0.46\linewidth]{Notebooks/PlotLumi/1TeV/ratios/Zp-Zp.pdf}
%%%\end{figure}
%%
%%%%\begin{figure}[!t]
%%%%\includegraphics[width=0.46\linewidth]{Notebooks/PlotLumi/30TeV/lumis/WmL-WpL.pdf}
%%%%\includegraphics[width=0.46\linewidth]{Notebooks/PlotLumi/30TeV/lumis/WmL-WpT.pdf}
%%%%\includegraphics[width=0.46\linewidth]{Notebooks/PlotLumi/30TeV/lumis/WmT-WpL.pdf}
%%%%\includegraphics[width=0.46\linewidth]{Notebooks/PlotLumi/30TeV/lumis/WmT-WpT.pdf}
%%%%\includegraphics[width=0.46\linewidth]{Notebooks/PlotLumi/30TeV/lumis/Wmm-Wpm.pdf}
%%%%\includegraphics[width=0.46\linewidth]{Notebooks/PlotLumi/30TeV/lumis/Wmm-Wpp.pdf}
%%%%\end{figure}
%%%
%%%%\begin{figure}[!t]
%%%%\includegraphics[width=0.46\linewidth]{Notebooks/PlotLumi/30TeV/lumis/Wmp-Wpm.pdf}
%%%%\includegraphics[width=0.46\linewidth]{Notebooks/PlotLumi/30TeV/lumis/Wmp-Wpp.pdf}
%%%%\includegraphics[width=0.46\linewidth]{Notebooks/PlotLumi/30TeV/lumis/ZL-ZL.pdf}
%%%%\includegraphics[width=0.46\linewidth]{Notebooks/PlotLumi/30TeV/lumis/ZL-ZT.pdf}
%%%%\includegraphics[width=0.46\linewidth]{Notebooks/PlotLumi/30TeV/lumis/ZT-ZL.pdf}
%%%%\includegraphics[width=0.46\linewidth]{Notebooks/PlotLumi/30TeV/lumis/ZT-ZT.pdf}
%%%%\end{figure}
%%%
%%%%\begin{figure}[!t]
%%%%\includegraphics[width=0.46\linewidth]{Notebooks/PlotLumi/30TeV/lumis/Zm-Zm.pdf}
%%%%\includegraphics[width=0.46\linewidth]{Notebooks/PlotLumi/30TeV/lumis/Zp-Zm.pdf}
%%%%\includegraphics[width=0.46\linewidth]{Notebooks/PlotLumi/30TeV/lumis/Zp-Zp.pdf}
%%%%\end{figure}
%%%
%%
%%\clearpage
%%\subsection{30 TeV}
%%
%%%\begin{figure}[!t]
%%%\includegraphics[width=0.46\linewidth]{Notebooks/PlotLumi/30TeV/lumis/plotWWpolRandL.pdf}
%%%\includegraphics[width=0.46\linewidth]{Notebooks/PlotLumi/30TeV/lumis/plotWWpolTand0.pdf}
%%%\includegraphics[width=0.46\linewidth]{Notebooks/PlotLumi/30TeV/lumis/plotWmWpandWpWm.pdf}
%%%\includegraphics[width=0.46\linewidth]{Notebooks/PlotLumi/30TeV/lumis/plotZZpolRandL.pdf}
%%%\includegraphics[width=0.46\linewidth]{Notebooks/PlotLumi/30TeV/lumis/plotZZpolTand0.pdf}
%%%\end{figure}
%%
%%%\begin{figure}
%%%\includegraphics[width=0.46\linewidth]{Notebooks/PlotLumi/30TeV/lumis/plotgammaWZ.pdf}
%%%\end{figure}
%%
%%\clearpage
%%\subsubsection{Comparisons with EVAs}
%%
%%
%%%\begin{figure}[!t]
%%%\includegraphics[width=0.46\linewidth]{Notebooks/PlotLumi/30TeV/ratios/WmL-WpL.pdf}
%%%\includegraphics[width=0.46\linewidth]{Notebooks/PlotLumi/30TeV/ratios/WmL-WpT.pdf}
%%%\includegraphics[width=0.46\linewidth]{Notebooks/PlotLumi/30TeV/ratios/WmT-WpL.pdf}
%%%\includegraphics[width=0.46\linewidth]{Notebooks/PlotLumi/30TeV/ratios/WmT-WpT.pdf}
%%%\includegraphics[width=0.46\linewidth]{Notebooks/PlotLumi/30TeV/ratios/Wmm-Wpm.pdf}
%%%\includegraphics[width=0.46\linewidth]{Notebooks/PlotLumi/30TeV/ratios/Wmm-Wpp.pdf}
%%%\end{figure}
%%
%%%\begin{figure}[!t]
%%%\includegraphics[width=0.46\linewidth]{Notebooks/PlotLumi/30TeV/ratios/Wmp-Wpm.pdf}
%%%\includegraphics[width=0.46\linewidth]{Notebooks/PlotLumi/30TeV/ratios/Wmp-Wpp.pdf}
%%%\includegraphics[width=0.46\linewidth]{Notebooks/PlotLumi/30TeV/ratios/ZL-ZL.pdf}
%%%\includegraphics[width=0.46\linewidth]{Notebooks/PlotLumi/30TeV/ratios/ZL-ZT.pdf}
%%%\includegraphics[width=0.46\linewidth]{Notebooks/PlotLumi/30TeV/ratios/ZT-ZL.pdf}
%%%\includegraphics[width=0.46\linewidth]{Notebooks/PlotLumi/30TeV/ratios/ZT-ZT.pdf}
%%%\end{figure}
%%
%%%\begin{figure}[!t]
%%%\includegraphics[width=0.46\linewidth]{Notebooks/PlotLumi/30TeV/ratios/Zm-Zm.pdf}
%%%\includegraphics[width=0.46\linewidth]{Notebooks/PlotLumi/30TeV/ratios/Zp-Zm.pdf}
%%%\includegraphics[width=0.46\linewidth]{Notebooks/PlotLumi/30TeV/ratios/Zp-Zp.pdf}
%%%\end{figure}
%%
%%%%\begin{figure}[!t]
%%%%\includegraphics[width=0.46\linewidth]{Notebooks/PlotLumi/3TeV/lumis/WmL-WpL.pdf}
%%%%\includegraphics[width=0.46\linewidth]{Notebooks/PlotLumi/3TeV/lumis/WmL-WpT.pdf}
%%%%\includegraphics[width=0.46\linewidth]{Notebooks/PlotLumi/3TeV/lumis/WmT-WpL.pdf}
%%%%\includegraphics[width=0.46\linewidth]{Notebooks/PlotLumi/3TeV/lumis/WmT-WpT.pdf}
%%%%\includegraphics[width=0.46\linewidth]{Notebooks/PlotLumi/3TeV/lumis/Wmm-Wpm.pdf}
%%%%\includegraphics[width=0.46\linewidth]{Notebooks/PlotLumi/3TeV/lumis/Wmm-Wpp.pdf}
%%%%\end{figure}
%%%
%%%%\begin{figure}[!t]
%%%%\includegraphics[width=0.46\linewidth]{Notebooks/PlotLumi/3TeV/lumis/Wmp-Wpm.pdf}
%%%%\includegraphics[width=0.46\linewidth]{Notebooks/PlotLumi/3TeV/lumis/Wmp-Wpp.pdf}
%%%%\includegraphics[width=0.46\linewidth]{Notebooks/PlotLumi/3TeV/lumis/ZL-ZL.pdf}
%%%%\includegraphics[width=0.46\linewidth]{Notebooks/PlotLumi/3TeV/lumis/ZL-ZT.pdf}
%%%%\includegraphics[width=0.46\linewidth]{Notebooks/PlotLumi/3TeV/lumis/ZT-ZL.pdf}
%%%%\includegraphics[width=0.46\linewidth]{Notebooks/PlotLumi/3TeV/lumis/ZT-ZT.pdf}
%%%%\end{figure}
%%%
%%%%\begin{figure}[!t]
%%%%\includegraphics[width=0.46\linewidth]{Notebooks/PlotLumi/3TeV/lumis/Zm-Zm.pdf}
%%%%\includegraphics[width=0.46\linewidth]{Notebooks/PlotLumi/3TeV/lumis/Zp-Zm.pdf}
%%%%\includegraphics[width=0.46\linewidth]{Notebooks/PlotLumi/3TeV/lumis/Zp-Zp.pdf}
%%%%\end{figure}
%%
%%\clearpage
%%\subsection{3 TeV}
%%
%%%\begin{figure}[!t]
%%%\includegraphics[width=0.46\linewidth]{Notebooks/PlotLumi/3TeV/lumis/plotWWpolRandL.pdf}
%%%\includegraphics[width=0.46\linewidth]{Notebooks/PlotLumi/3TeV/lumis/plotWWpolTand0.pdf}
%%%\includegraphics[width=0.46\linewidth]{Notebooks/PlotLumi/3TeV/lumis/plotWmWpandWpWm.pdf}
%%%\includegraphics[width=0.46\linewidth]{Notebooks/PlotLumi/3TeV/lumis/plotZZpolRandL.pdf}
%%%\includegraphics[width=0.46\linewidth]{Notebooks/PlotLumi/3TeV/lumis/plotZZpolTand0.pdf}
%%%\end{figure}
%%
%%%\begin{figure}
%%%\includegraphics[width=0.46\linewidth]{Notebooks/PlotLumi/3TeV/lumis/plotgammaWZ.pdf}
%%%\end{figure}
%%
%%
%%\clearpage
%%\subsubsection{Comparisons with EVAs}
%%
%%
%%
%%%\begin{figure}[!t]
%%%\includegraphics[width=0.46\linewidth]{Notebooks/PlotLumi/3TeV/ratios/WmL-WpL.pdf}
%%%\includegraphics[width=0.46\linewidth]{Notebooks/PlotLumi/3TeV/ratios/WmL-WpT.pdf}
%%%\includegraphics[width=0.46\linewidth]{Notebooks/PlotLumi/3TeV/ratios/WmT-WpL.pdf}
%%%\includegraphics[width=0.46\linewidth]{Notebooks/PlotLumi/3TeV/ratios/WmT-WpT.pdf}
%%%\includegraphics[width=0.46\linewidth]{Notebooks/PlotLumi/3TeV/ratios/Wmm-Wpm.pdf}
%%%\includegraphics[width=0.46\linewidth]{Notebooks/PlotLumi/3TeV/ratios/Wmm-Wpp.pdf}
%%%\end{figure}
%%
%%%\begin{figure}[!t]
%%%\includegraphics[width=0.46\linewidth]{Notebooks/PlotLumi/3TeV/ratios/Wmp-Wpm.pdf}
%%%\includegraphics[width=0.46\linewidth]{Notebooks/PlotLumi/3TeV/ratios/Wmp-Wpp.pdf}
%%%\includegraphics[width=0.46\linewidth]{Notebooks/PlotLumi/3TeV/ratios/ZL-ZL.pdf}
%%%\includegraphics[width=0.46\linewidth]{Notebooks/PlotLumi/3TeV/ratios/ZL-ZT.pdf}
%%%\includegraphics[width=0.46\linewidth]{Notebooks/PlotLumi/3TeV/ratios/ZT-ZL.pdf}
%%%\includegraphics[width=0.46\linewidth]{Notebooks/PlotLumi/3TeV/ratios/ZT-ZT.pdf}
%%%\end{figure}
%%
%%%\begin{figure}[!t]
%%%\includegraphics[width=0.46\linewidth]{Notebooks/PlotLumi/3TeV/ratios/Zm-Zm.pdf}
%%%\includegraphics[width=0.46\linewidth]{Notebooks/PlotLumi/3TeV/ratios/Zp-Zm.pdf}
%%%\includegraphics[width=0.46\linewidth]{Notebooks/PlotLumi/3TeV/ratios/Zp-Zp.pdf}
%%%\end{figure}
%%
%%
%%
%%\clearpage
%%\section{PDFs vs EVA}
%%
%%\subsection{All together, for $\mu+$ and $\mu-$ but same partons}
%%
%%%\begin{figure}[!t]
%%%\includegraphics[width=0.46\linewidth]{Notebooks/PlotPDFs/alltogether/10TeV_mu+scaleQ.pdf}
%%%\includegraphics[width=0.46\linewidth]{Notebooks/PlotPDFs/alltogether/10TeV_mu+scaleQsqrtx.pdf}
%%%\includegraphics[width=0.46\linewidth]{Notebooks/PlotPDFs/alltogether/10TeV_mu+scaleQx.pdf}
%%%\end{figure}
%%
%%%\begin{figure}[!t]
%%%\includegraphics[width=0.46\linewidth]{Notebooks/PlotPDFs/alltogether/10TeV_mu-scaleQ.pdf}
%%%\includegraphics[width=0.46\linewidth]{Notebooks/PlotPDFs/alltogether/10TeV_mu-scaleQsqrtx.pdf}
%%%\includegraphics[width=0.46\linewidth]{Notebooks/PlotPDFs/alltogether/10TeV_mu-scaleQx.pdf}
%%%\end{figure}
%%
%%%\begin{figure}[!t]
%%%\includegraphics[width=0.46\linewidth]{Notebooks/PlotPDFs/alltogether/1TeV_mu+scaleQ.pdf}
%%%\includegraphics[width=0.46\linewidth]{Notebooks/PlotPDFs/alltogether/1TeV_mu+scaleQsqrtx.pdf}
%%%\includegraphics[width=0.46\linewidth]{Notebooks/PlotPDFs/alltogether/1TeV_mu+scaleQx.pdf}
%%%\end{figure}
%%
%%
%%%\begin{figure}[!t]
%%%\includegraphics[width=0.46\linewidth]{Notebooks/PlotPDFs/alltogether/1TeV_mu-scaleQ.pdf}
%%%\includegraphics[width=0.46\linewidth]{Notebooks/PlotPDFs/alltogether/1TeV_mu-scaleQsqrtx.pdf}
%%%\includegraphics[width=0.46\linewidth]{Notebooks/PlotPDFs/alltogether/1TeV_mu-scaleQx.pdf}
%%%\end{figure}
%%
%%%\begin{figure}[!t]
%%%\includegraphics[width=0.46\linewidth]{Notebooks/PlotPDFs/alltogether/30TeV_mu+scaleQ.pdf}
%%%\includegraphics[width=0.46\linewidth]{Notebooks/PlotPDFs/alltogether/30TeV_mu+scaleQsqrtx.pdf}
%%%\includegraphics[width=0.46\linewidth]{Notebooks/PlotPDFs/alltogether/30TeV_mu+scaleQx.pdf}
%%%\end{figure}
%%
%%%\begin{figure}[!t]
%%%\includegraphics[width=0.46\linewidth]{Notebooks/PlotPDFs/alltogether/30TeV_mu-scaleQ.pdf}
%%%\includegraphics[width=0.46\linewidth]{Notebooks/PlotPDFs/alltogether/30TeV_mu-scaleQsqrtx.pdf}
%%%\includegraphics[width=0.46\linewidth]{Notebooks/PlotPDFs/alltogether/30TeV_mu-scaleQx.pdf}
%%%\end{figure}
%%
%%%\begin{figure}[!t]
%%%\includegraphics[width=0.46\linewidth]{Notebooks/PlotPDFs/alltogether/3TeV_mu+scaleQ.pdf}
%%%\includegraphics[width=0.46\linewidth]{Notebooks/PlotPDFs/alltogether/3TeV_mu+scaleQsqrtx.pdf}
%%%\includegraphics[width=0.46\linewidth]{Notebooks/PlotPDFs/alltogether/3TeV_mu+scaleQx.pdf}
%%%\end{figure}
%%
%%%\begin{figure}[!t]
%%%\includegraphics[width=0.46\linewidth]{Notebooks/PlotPDFs/alltogether/3TeV_mu-scaleQ.pdf}
%%%\includegraphics[width=0.46\linewidth]{Notebooks/PlotPDFs/alltogether/3TeV_mu-scaleQsqrtx.pdf}
%%%\includegraphics[width=0.46\linewidth]{Notebooks/PlotPDFs/alltogether/3TeV_mu-scaleQx.pdf}
%%%\end{figure}
%%
%%
%%
%%\clearpage
%%\subsection{Individuals PDF vs EVAs 10 TeV}
%%
%%%\begin{figure}[!t]
%%%\includegraphics[width=0.46\linewidth]{Notebooks/PlotPDFs/ratios/10TeV/W-_0_Q.pdf}
%%%\includegraphics[width=0.46\linewidth]{Notebooks/PlotPDFs/ratios/10TeV/W-_0_Qsqrtx.pdf}
%%%\includegraphics[width=0.46\linewidth]{Notebooks/PlotPDFs/ratios/10TeV/W-_0_Qx.pdf}
%%%\end{figure}
%%
%%%\begin{figure}[!t]
%%%\includegraphics[width=0.46\linewidth]{Notebooks/PlotPDFs/ratios/10TeV/W-_L_Q.pdf}
%%%\includegraphics[width=0.46\linewidth]{Notebooks/PlotPDFs/ratios/10TeV/W-_L_Qsqrtx.pdf}
%%%\includegraphics[width=0.46\linewidth]{Notebooks/PlotPDFs/ratios/10TeV/W-_L_Qx.pdf}
%%%\end{figure}
%%
%%%\begin{figure}[!t]
%%%\includegraphics[width=0.46\linewidth]{Notebooks/PlotPDFs/ratios/10TeV/W-_R_Q.pdf}
%%%\includegraphics[width=0.46\linewidth]{Notebooks/PlotPDFs/ratios/10TeV/W-_R_Qsqrtx.pdf}
%%%\includegraphics[width=0.46\linewidth]{Notebooks/PlotPDFs/ratios/10TeV/W-_R_Qx.pdf}
%%%\end{figure}
%%
%%%\begin{figure}[!t]
%%%\includegraphics[width=0.46\linewidth]{Notebooks/PlotPDFs/ratios/10TeV/W-_T_Q.pdf}
%%%\includegraphics[width=0.46\linewidth]{Notebooks/PlotPDFs/ratios/10TeV/W-_T_Qsqrtx.pdf}
%%%\includegraphics[width=0.46\linewidth]{Notebooks/PlotPDFs/ratios/10TeV/W-_T_Qx.pdf}
%%%\end{figure}
%%
%%%\begin{figure}[!t]
%%%\includegraphics[width=0.46\linewidth]{Notebooks/PlotPDFs/ratios/10TeV/Z_0_Q.pdf}
%%%\includegraphics[width=0.46\linewidth]{Notebooks/PlotPDFs/ratios/10TeV/Z_0_Qsqrtx.pdf}
%%%\includegraphics[width=0.46\linewidth]{Notebooks/PlotPDFs/ratios/10TeV/Z_0_Qx.pdf}
%%%\end{figure}
%%
%%%\begin{figure}[!t]
%%%\includegraphics[width=0.46\linewidth]{Notebooks/PlotPDFs/ratios/10TeV/Z_L-_Q.pdf}
%%%\includegraphics[width=0.46\linewidth]{Notebooks/PlotPDFs/ratios/10TeV/Z_L-_Qsqrtx.pdf}
%%%\includegraphics[width=0.46\linewidth]{Notebooks/PlotPDFs/ratios/10TeV/Z_L-_Qx.pdf}
%%%\end{figure}
%%
%%%\begin{figure}[!t]
%%%\includegraphics[width=0.46\linewidth]{Notebooks/PlotPDFs/ratios/10TeV/Z_R_Q.pdf}
%%%\includegraphics[width=0.46\linewidth]{Notebooks/PlotPDFs/ratios/10TeV/Z_R_Qsqrtx.pdf}
%%%\includegraphics[width=0.46\linewidth]{Notebooks/PlotPDFs/ratios/10TeV/Z_R_Qx.pdf}
%%%\end{figure}
%%
%%%\begin{figure}[!t]
%%%\includegraphics[width=0.46\linewidth]{Notebooks/PlotPDFs/ratios/10TeV/Z_T_Q.pdf}
%%%\includegraphics[width=0.46\linewidth]{Notebooks/PlotPDFs/ratios/10TeV/Z_T_Qsqrtx.pdf}
%%%\includegraphics[width=0.46\linewidth]{Notebooks/PlotPDFs/ratios/10TeV/Z_T_Qx.pdf}
%%%\end{figure}
%%
%%%\begin{figure}[!t]
%%%\includegraphics[width=0.46\linewidth]{Notebooks/PlotPDFs/ratios/1TeV/W-_0_Q.pdf}
%%%\includegraphics[width=0.46\linewidth]{Notebooks/PlotPDFs/ratios/1TeV/W-_0_Qsqrtx.pdf}
%%%\includegraphics[width=0.46\linewidth]{Notebooks/PlotPDFs/ratios/1TeV/W-_0_Qx.pdf}
%%%\end{figure}
%%
%%\clearpage
%%\subsection{Individuals PDF vs EVAs 1 TeV}
%%
%%%\begin{figure}[!t]
%%%\includegraphics[width=0.46\linewidth]{Notebooks/PlotPDFs/ratios/1TeV/W-_L_Q.pdf}
%%%\includegraphics[width=0.46\linewidth]{Notebooks/PlotPDFs/ratios/1TeV/W-_L_Qsqrtx.pdf}
%%%\includegraphics[width=0.46\linewidth]{Notebooks/PlotPDFs/ratios/1TeV/W-_L_Qx.pdf}
%%%\end{figure}
%%
%%%\begin{figure}[!t]
%%%\includegraphics[width=0.46\linewidth]{Notebooks/PlotPDFs/ratios/1TeV/W-_R_Q.pdf}
%%%\includegraphics[width=0.46\linewidth]{Notebooks/PlotPDFs/ratios/1TeV/W-_R_Qsqrtx.pdf}
%%%\includegraphics[width=0.46\linewidth]{Notebooks/PlotPDFs/ratios/1TeV/W-_R_Qx.pdf}
%%%\end{figure}
%%
%%%\begin{figure}[!t]
%%%\includegraphics[width=0.46\linewidth]{Notebooks/PlotPDFs/ratios/1TeV/W-_T_Q.pdf}
%%%\includegraphics[width=0.46\linewidth]{Notebooks/PlotPDFs/ratios/1TeV/W-_T_Qsqrtx.pdf}
%%%\includegraphics[width=0.46\linewidth]{Notebooks/PlotPDFs/ratios/1TeV/W-_T_Qx.pdf}
%%%\end{figure}
%%
%%%\begin{figure}[!t]
%%%\includegraphics[width=0.46\linewidth]{Notebooks/PlotPDFs/ratios/1TeV/Z_0_Q.pdf}
%%%\includegraphics[width=0.46\linewidth]{Notebooks/PlotPDFs/ratios/1TeV/Z_0_Qsqrtx.pdf}
%%%\includegraphics[width=0.46\linewidth]{Notebooks/PlotPDFs/ratios/1TeV/Z_0_Qx.pdf}
%%%\end{figure}
%%
%%%\begin{figure}[!t]
%%%\includegraphics[width=0.46\linewidth]{Notebooks/PlotPDFs/ratios/1TeV/Z_L-_Q.pdf}
%%%\includegraphics[width=0.46\linewidth]{Notebooks/PlotPDFs/ratios/1TeV/Z_L-_Qsqrtx.pdf}
%%%\includegraphics[width=0.46\linewidth]{Notebooks/PlotPDFs/ratios/1TeV/Z_L-_Qx.pdf}
%%%\end{figure}
%%
%%%\begin{figure}[!t]
%%%\includegraphics[width=0.46\linewidth]{Notebooks/PlotPDFs/ratios/1TeV/Z_R_Q.pdf}
%%%\includegraphics[width=0.46\linewidth]{Notebooks/PlotPDFs/ratios/1TeV/Z_R_Qsqrtx.pdf}
%%%\includegraphics[width=0.46\linewidth]{Notebooks/PlotPDFs/ratios/1TeV/Z_R_Qx.pdf}
%%%\end{figure}
%%
%%%\begin{figure}[!t]
%%%\includegraphics[width=0.46\linewidth]{Notebooks/PlotPDFs/ratios/1TeV/Z_T_Q.pdf}
%%%\includegraphics[width=0.46\linewidth]{Notebooks/PlotPDFs/ratios/1TeV/Z_T_Qsqrtx.pdf}
%%%\includegraphics[width=0.46\linewidth]{Notebooks/PlotPDFs/ratios/1TeV/Z_T_Qx.pdf}
%%%\end{figure}
%%
%%\clearpage
%%\subsection{Individuals PDF vs EVAs 30 TeV}
%%
%%%\begin{figure}[!t]
%%%\includegraphics[width=0.46\linewidth]{Notebooks/PlotPDFs/ratios/30TeV/W-_0_Q.pdf}
%%%\includegraphics[width=0.46\linewidth]{Notebooks/PlotPDFs/ratios/30TeV/W-_0_Qsqrtx.pdf}
%%%\includegraphics[width=0.46\linewidth]{Notebooks/PlotPDFs/ratios/30TeV/W-_0_Qx.pdf}
%%%\end{figure}
%%
%%%\begin{figure}[!t]
%%%\includegraphics[width=0.46\linewidth]{Notebooks/PlotPDFs/ratios/30TeV/W-_L_Q.pdf}
%%%\includegraphics[width=0.46\linewidth]{Notebooks/PlotPDFs/ratios/30TeV/W-_L_Qsqrtx.pdf}
%%%\includegraphics[width=0.46\linewidth]{Notebooks/PlotPDFs/ratios/30TeV/W-_L_Qx.pdf}
%%%\end{figure}
%%
%%%\begin{figure}[!t]
%%%\includegraphics[width=0.46\linewidth]{Notebooks/PlotPDFs/ratios/30TeV/W-_R_Q.pdf}
%%%\includegraphics[width=0.46\linewidth]{Notebooks/PlotPDFs/ratios/30TeV/W-_R_Qsqrtx.pdf}
%%%\includegraphics[width=0.46\linewidth]{Notebooks/PlotPDFs/ratios/30TeV/W-_R_Qx.pdf}
%%%\end{figure}
%%
%%
%%%\begin{figure}[!t]
%%%\includegraphics[width=0.46\linewidth]{Notebooks/PlotPDFs/ratios/30TeV/W-_T_Q.pdf}
%%%\includegraphics[width=0.46\linewidth]{Notebooks/PlotPDFs/ratios/30TeV/W-_T_Qsqrtx.pdf}
%%%\includegraphics[width=0.46\linewidth]{Notebooks/PlotPDFs/ratios/30TeV/W-_T_Qx.pdf}
%%%\end{figure}
%%
%%%\begin{figure}[!t]
%%%\includegraphics[width=0.46\linewidth]{Notebooks/PlotPDFs/ratios/30TeV/Z_0_Q.pdf}
%%%\includegraphics[width=0.46\linewidth]{Notebooks/PlotPDFs/ratios/30TeV/Z_0_Qsqrtx.pdf}
%%%\includegraphics[width=0.46\linewidth]{Notebooks/PlotPDFs/ratios/30TeV/Z_0_Qx.pdf}
%%%\end{figure}
%%
%%%\begin{figure}[!t]
%%%\includegraphics[width=0.46\linewidth]{Notebooks/PlotPDFs/ratios/30TeV/Z_L-_Q.pdf}
%%%\includegraphics[width=0.46\linewidth]{Notebooks/PlotPDFs/ratios/30TeV/Z_L-_Qsqrtx.pdf}
%%%\includegraphics[width=0.46\linewidth]{Notebooks/PlotPDFs/ratios/30TeV/Z_L-_Qx.pdf}
%%%\end{figure}
%%
%%%\begin{figure}[!t]
%%%\includegraphics[width=0.46\linewidth]{Notebooks/PlotPDFs/ratios/30TeV/Z_R_Q.pdf}
%%%\includegraphics[width=0.46\linewidth]{Notebooks/PlotPDFs/ratios/30TeV/Z_R_Qsqrtx.pdf}
%%%\includegraphics[width=0.46\linewidth]{Notebooks/PlotPDFs/ratios/30TeV/Z_R_Qx.pdf}
%%%\end{figure}
%%
%%%\begin{figure}[!t]
%%%\includegraphics[width=0.46\linewidth]{Notebooks/PlotPDFs/ratios/30TeV/Z_T_Q.pdf}
%%%\includegraphics[width=0.46\linewidth]{Notebooks/PlotPDFs/ratios/30TeV/Z_T_Qsqrtx.pdf}
%%%\includegraphics[width=0.46\linewidth]{Notebooks/PlotPDFs/ratios/30TeV/Z_T_Qx.pdf}
%%%\end{figure}
%%
%%%\begin{figure}[!t]
%%%\includegraphics[width=0.46\linewidth]{Notebooks/PlotPDFs/ratios/3TeV/W-_0_Q.pdf}
%%%\includegraphics[width=0.46\linewidth]{Notebooks/PlotPDFs/ratios/3TeV/W-_0_Qsqrtx.pdf}
%%%\includegraphics[width=0.46\linewidth]{Notebooks/PlotPDFs/ratios/3TeV/W-_0_Qx.pdf}
%%%\end{figure}
%%
%%%\begin{figure}[!t]
%%%\includegraphics[width=0.46\linewidth]{Notebooks/PlotPDFs/ratios/3TeV/W-_L_Q.pdf}
%%%\includegraphics[width=0.46\linewidth]{Notebooks/PlotPDFs/ratios/3TeV/W-_L_Qsqrtx.pdf}
%%%\includegraphics[width=0.46\linewidth]{Notebooks/PlotPDFs/ratios/3TeV/W-_L_Qx.pdf}
%%%\end{figure}
%%
%%%\begin{figure}[!t]
%%%\includegraphics[width=0.46\linewidth]{Notebooks/PlotPDFs/ratios/3TeV/W-_R_Q.pdf}
%%%\includegraphics[width=0.46\linewidth]{Notebooks/PlotPDFs/ratios/3TeV/W-_R_Qsqrtx.pdf}
%%%\includegraphics[width=0.46\linewidth]{Notebooks/PlotPDFs/ratios/3TeV/W-_R_Qx.pdf}
%%%\end{figure}
%%
%%\clearpage
%%\subsection{Individuals PDF vs EVAs 3 TeV}
%%
%%%\begin{figure}[!t]
%%%\includegraphics[width=0.46\linewidth]{Notebooks/PlotPDFs/ratios/3TeV/W-_T_Q.pdf}
%%%\includegraphics[width=0.46\linewidth]{Notebooks/PlotPDFs/ratios/3TeV/W-_T_Qsqrtx.pdf}
%%%\includegraphics[width=0.46\linewidth]{Notebooks/PlotPDFs/ratios/3TeV/W-_T_Qx.pdf}
%%%\end{figure}
%%
%%%\begin{figure}[!t]
%%%\includegraphics[width=0.46\linewidth]{Notebooks/PlotPDFs/ratios/3TeV/Z_0_Q.pdf}
%%%\includegraphics[width=0.46\linewidth]{Notebooks/PlotPDFs/ratios/3TeV/Z_0_Qsqrtx.pdf}
%%%\includegraphics[width=0.46\linewidth]{Notebooks/PlotPDFs/ratios/3TeV/Z_0_Qx.pdf}
%%%\end{figure}
%%
%%%\begin{figure}[!t]
%%%\includegraphics[width=0.46\linewidth]{Notebooks/PlotPDFs/ratios/3TeV/Z_L-_Q.pdf}
%%%\includegraphics[width=0.46\linewidth]{Notebooks/PlotPDFs/ratios/3TeV/Z_L-_Qsqrtx.pdf}
%%%\includegraphics[width=0.46\linewidth]{Notebooks/PlotPDFs/ratios/3TeV/Z_L-_Qx.pdf}
%%%\end{figure}
%%
%%%\begin{figure}[!t]
%%%\includegraphics[width=0.46\linewidth]{Notebooks/PlotPDFs/ratios/3TeV/Z_R_Q.pdf}
%%%\includegraphics[width=0.46\linewidth]{Notebooks/PlotPDFs/ratios/3TeV/Z_R_Qsqrtx.pdf}
%%%\includegraphics[width=0.46\linewidth]{Notebooks/PlotPDFs/ratios/3TeV/Z_R_Qx.pdf}
%%%\end{figure}
%%
%%%\begin{figure}[!t]
%%%\includegraphics[width=0.46\linewidth]{Notebooks/PlotPDFs/ratios/3TeV/Z_T_Q.pdf}
%%%\includegraphics[width=0.46\linewidth]{Notebooks/PlotPDFs/ratios/3TeV/Z_T_Qsqrtx.pdf}
%%%\includegraphics[width=0.46\linewidth]{Notebooks/PlotPDFs/ratios/3TeV/Z_T_Qx.pdf}
%%%\end{figure}
%%
%%
\bibliographystyle{JHEP}
\bibliography{bibliography}

\end{document}